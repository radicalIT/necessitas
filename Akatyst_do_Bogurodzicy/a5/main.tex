\documentclass[10pt]{article}

% Ustawienia strony
\usepackage[a5paper,top=1.7cm, left=2cm, right=2cm, bottom=2.5cm]{geometry}
\usepackage[polish]{babel}
\usepackage{fontspec}
\usepackage{stackengine}
\usepackage{csquotes}
\usepackage{xcolor}
\usepackage{pagecolor}
\usepackage{tikz}
\usepackage{eso-pic}
\usepackage{fancyhdr}
\usepackage{graphicx}
\usepackage{setspace}
\usepackage{array}
\usepackage{titlesec}
\usepackage{ragged2e}
\usepackage{microtype}
\usepackage{enumitem}

\titleformat{\chapter}
{\normalfont\Large\bfseries}{\thechapter.}{1em}{}

\titleformat{\section}
{\center\normalfont\large\bfseries}{\thesection.}{1em}{}

\titleformat{\subsection}
{\center\normalfont\normalsize\bfseries}{\thesubsection.}{1em}{}
\titleformat{\subsubsection}
{\normalfont\small\bfseries}{\thesubsection.}{1em}{}

\pagestyle{plain}

% \titleformat{\chapter}[display]
%   {\normalsize \huge  \color{black}}%
%   {\flushright\normalsize \color{RoyalRed}%
%    \MakeUppercase{\chaptertitlename}\hspace{1ex}%
%    {\fontfamily{mdugm}\fontsize{60}{60}\selectfont\thechapter}}%
%   {10 pt}%
%   {\bfseries\huge}%
\makeatletter
\def\@makechapterhead#1{%
  {\parindent \z@ \raggedright \normalfont
      % \ifnum \c@secnumdepth >\m@ne
      %     \Large\bfseries\thechapter.\space
      % \fi
      \interlinepenalty\@M
      \huge \bfseries #1\par\nobreak
      \vskip 40\p@
    }}
\makeatother

\setcounter{secnumdepth}{0}

\renewcommand{\cleardoublepage}{\clearpage}
% \renewcommand{\thesection}{abc}
% \let\cleardoublepage=\clearpage

\newgeometry{
  top=20mm,
  bottom=20mm,
  outer=12mm,
  inner=15mm,
}

\grechangestaffsize{15}
\gresetinitiallines{0}
\grechangestyle{annotation}{\small\bfseries}
\grechangestyle{initial}{\fontsize{32}{35}\mdseries}
\grechangedim{spaceabovelines}{2mm}{scalable}
\gresetheadercapture{sigla}{grecommentary[2mm]}{}


% \titleformat{\subsection}
%   {\normalfont\normalsize\itshape}{\thesubsection.}{1em}{}

% \titleformat{\subsubsection}
%   {\normalfont\normalsize\itshape}{\thesubsubsection.}{1em}{}
\newcommand{\fakesc}[2]{\MakeUppercase{#1}\scalebox{0.75}{\MakeUppercase{#2}}}

\newcommand{\ikos}[1]{\fakesc{I}{kos} #1}
\newcommand{\kondakion}[1]{\fakesc{K}{ondakion} #1}

\newcommand{\inputIkos}[1]{\input{Parts/ikos_#1.tex}}
\newcommand{\inputKondakion}[1]{\input{Parts/kondakion_#1.tex}}

\newcommand{\ByzSym}[1]{\textcolor{darkgray}{\scalebox{1.3}{\bizfont\char"#1}}}

\newcommand{\init}[2]{
    \lettrine[lraise=0.1, nindent=0em, slope=-.5em]{\redortho{#1}}{#2}
}

\newcommand{\newTitle}[2]{
    \begin{center}
        {
            {\color{red}\fontsize{53}{53}\ortfont #1}\\
            {\Large\wfont #2}
        }
    \end{center}
}

\newcommand{\redortho}[1]{{\textcolor{red}{\ortfont#1}}}

\newcommand{\comm}[1]{\textcolor{red}{#1}}
\newcommand{\sep}{\vspace*{1cm}}
\newcommand{\sepSmall}{\vspace*{0.5cm}}
\newcommand{\sepBig}{\vspace*{1.5cm}}

\newcommand{\nl}{\vspace*{-1mm}\\\hspace*{8mm}}
\newcommand{\cutLine}{\vspace*{-\baselineskip}}

\newcommand{\witaj}{\noindent{\fontsize{17}{17}\comm{\wfont W}}itaj}

\newcommand{\allelujaiii}{
    \tertia{\fontsize{17}{17}\comm{\wfont A}l}leluja, \comm{\Large\wfont a}lle{\normalsize\petasti{lu}ja}, \comm{\Large\wfont a}lleluja.
}
\newcommand{\alleluja}{
    \apostrofos{\fontsize{13}{13}\comm{\wfont A}l}\!\oligon{le}\!\!\oligon[3pt]{lu}\!ja.
}
\newcommand{\witajOblubienico}{\noindent\oligon{\witaj}, Oblubie\petasti{ni}co Dziewicza.}

\newcommand{\V}[1]{\textcolor{red}{$\mathbf{\not{\mkern -3mu \mathrm{V}}}.$}~#1}
\newcommand{\R}[1]{\textcolor{red}{$\mathbf{\not{\mkern -3mu \mathrm{R}}}.$}~#1}

% Znaki bizantyjskie
\newcommand{\ison}[1]{\stackon[2pt]{#1}{\ByzSym{1D046}}}
\newcommand{\oligon}[2][0pt]{\stackon[#1]{#2}{\vspace{#1}\ByzSym{1D047}}}
\newcommand{\apostrofos}[2][2pt]{\stackon[#1]{#2}{\ByzSym{1D051}}}
\newcommand{\tertia}[1]{\stackon[-1pt]{#1}{\stackon[-2pt]{\hspace{5pt}\ByzSym{1D0F1}}{\ByzSym{1D047}}}}
\newcommand{\elaphron}[1]{\stackon[4pt]{#1}{\rotatebox{180}{\ByzSym{1D049}}}}
\newcommand{\petasti}[1]{\stackon[2pt]{{\bf#1}}{\ByzSym{1D049}}}
\newcommand{\kentemata}[1]{\stackon[0pt]{#1}{\stackon[-4pt]{\ByzSym{1D046}}{\ByzSym{1D0F0}}}}
\newcommand{\pauza}{\textcolor{darkgray}{\raisebox{2pt}{\ \scalebox{1.5}{’}\ }}}

\begin{document}

\begin{center}
    \includegraphics[width=0.8\textwidth]{Sources/nmp.pdf}    
\end{center}
\thispagestyle{empty}
\vfill
\newTitle{Akatyst}{ku czci Bogurodzicy}

\clearpage
\section*{\fakesc{W}{prowadzenie}}
\small
Celem tego opracowania było przygotowanie zapisu możliwie intuicyjnego. Podczas śpiewu wystarczy zwracać uwagę przede wszystkim na pogrubione sylaby. Dla większej precyzji w notacji wprowadziłem również poniższe oznaczenia, zaczerpnięte z tradycji bizantyjskiej:
\begin{description}
    \item [\comm{Petastē}] \enquote{\scalebox{1.3}{\ByzSym{1D049}}}\hfill\\podniesienie głosu o sekundę (+1) pod akcentem, np.:\\\includegraphics{Notes/Legenda/petaste.png}
    \item [\comm{Oligon}] \enquote{\ByzSym{1D047}}\hfill\\podniesienie głosu o sekundę (+1), np.:\\\includegraphics{Notes/Legenda/oligon.png}
    \item [\comm{Ison z Kentēmatą}] \enquote{\scalebox{1.3}{\stackon[-3.5pt]{\ByzSym{1D046}}{\ByzSym{1D0F0}}}}\hfill\\przeciągnięcie głosu i podniesienie o sekundę (0+1), np.:\\\includegraphics{Notes/Legenda/ison_i_kentemata.png}
    \clearpage
    \item [\comm{Olígon z Kéntēmą}] \enquote{\raisebox{-3pt}{\scalebox{1.3}{\stackon[-2pt]{\hspace{5pt}\ByzSym{1D0F1}}{\ByzSym{1D047}}}}}\hfill\\podniesienie głosu o tercję (+2), np.:\\\includegraphics{Notes/Legenda/oligon_i_kentema.png}
    \item [\comm{Apóstrophos}] \enquote{\scalebox{1.6}{\ByzSym{1D051}}}\hfill\\obniżenie głosu o sekundę (-1), np.:\\\includegraphics{Notes/Legenda/aposrophos.png}
    \item [\comm{Elaphron}] \enquote{\scalebox{1.3}{\raisebox{7pt}{\rotatebox{180}{\ByzSym{1D049}}}}}\hfill\\obniżenie głosu o tercję (-2), np.:\\\includegraphics{Notes/Legenda/elaphron.png}
    \item [\comm{Pauza}] \enquote{\scalebox{1.2}{’}}\hfill\\pauza na oddech, np.:\\\includegraphics{Notes/Legenda/pauza.png}
\end{description}
\normalsize

\section*{\fakesc{R}{ozpoczęcie}}
\begin{center}
    \makebox[\linewidth]{%
        \includegraphics[width=1\textwidth]{Notes/Boże, wejrzyj_full.png}
    }
\end{center}

\vfill
\begin{center}
    \makebox[\linewidth]{%
        \includegraphics[width=0.22\textwidth]{Sources/Ryciny/cross3.png}
    }
\end{center}

\clearpage

\section*{\kondakion{1}}
\label{kon_1}
\inputKondakion{1}

\section*{\ikos{1}}
\label{ikos_1}
\inputIkos{1}

\section*{\kondakion{2}}
\inputKondakion{2}

\section*{\ikos{2}}
\inputIkos{2}

\section*{\kondakion{3}}
\inputKondakion{3}

\section*{\ikos{3}}
\inputIkos{3}

\section*{\kondakion{4}}
\inputKondakion{4}

\section*{\ikos{4}}
\inputIkos{4}

\section*{\kondakion{5}}
\inputKondakion{5}

\section*{\ikos{5}}
\inputIkos{5}

\section*{\kondakion{6}}
\inputKondakion{6}

\section*{\ikos{6}}
\inputIkos{6}

\section*{\kondakion{7}}
\inputKondakion{7}

\section*{\ikos{7}}
\inputIkos{7}

\section*{\kondakion{8}}
\inputKondakion{8}

\clearpage

\section*{\ikos{8}}
\inputIkos{8}

\section*{\kondakion{9}}
\inputKondakion{9}

\section*{\ikos{9}}
\inputIkos{9}

\section*{\kondakion{10}}
\inputKondakion{10}

\section*{\ikos{10}}
\inputIkos{10}

\section*{\kondakion{11}}
\inputKondakion{11}

\section*{\ikos{11}}
\inputIkos{11}

\section*{\kondakion{12}}
\inputKondakion{12}

\section*{\ikos{12}}
\inputIkos{12}

\section*{\kondakion{13}}
\inputKondakion{13}

\begin{itemize}[label={\includegraphics[width=0.4cm]{Sources/rotunda_symbol.png}}]
    \item \comm{Kondakion ten zwykle recytuje się trzykrotnie.}
    \item \comm{Według tradycji bizantyjsko-słowiańskiej na koniec dodaje się jeszcze \textcolor{black}{ikos 1} (s. \textcolor{black}{\pageref{ikos_1}}), zwłaszcza, gdy recytowany był cały akatyst.}
    \item \comm{Akatyst kończy się śpiewem \textcolor{black}{kondakionu 1} (s. \textcolor{black}{\pageref{kon_1}}).}
\end{itemize}

\vfill
\begin{center}
    \makebox[\linewidth]{%
        \includegraphics[width=0.7\textwidth]{Sources/Ryciny/narodzenie.png}
    }
\end{center}
\vfill

\clearpage
\section*{\fakesc{M}{odlitwa dziękczynna akatystu}}

Przyjmij o Potężna; najczystsza Pani i Królowo nasza, Boża Rodzicielko, te pochwalne hymny, które Tobie Jedynej z wdzięcznością śpiewamy my, słudzy Twoi niegodni. Wybrana spośród wszystkich pokoleń, nad wszystkie stworzenia nieba i ziemi wspanialszą się okazałaś. Za Twym udziałem Pan zastępów jest z nami, przez Ciebie poznaliśmy Syna Bożego i nasyciliśmy się świętym Jego Ciałem i Krwią najczystszą. Dlatego błogosławiona jesteś z pokolenia na pokolenie, o przez Boga wysławiona, wspanialsza od Cherubinów i czcigodniejsza od Serafinów. W tej oto godzinie, o pełna chwały Bogurodzico Najświętsza, nie ustawaj w modlitwie za nas, niegodne Twe sługi, byśmy od wszelkiej złej myśli i od wszelkiego nieszczęścia wolni byli i od wszelkiej szatańskiej złośliwości uchronieni. Aż do chwili ostatniej, wstawiając się za nami, zachowaj nas nienagannymi. Orędownictwem bowiem Twoim i opieką chronieni, Jedynemu w Trójcy Bogu Stwórcy wszechświata składamy chwałę, część, hołd i dziękczynienie, teraz i zawsze i na wieki wieków. Amen.
\vfill
\begin{center}
    \makebox[\linewidth]{%
        \includegraphics[width=0.1\textwidth]{Sources/Ryciny/cross.png}
    }
\end{center}
\end{document}
