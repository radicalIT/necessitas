\init{C}{zeka Dziewica z u\petasti{prag}\apostrofos{nie}niem\pauza na poznanie niepozna\apostrofos{wal}\apostrofos{ne}go\pauza i rzecze Bożemu słudze:\pauza \enquote{\oligon{Czyż} łono moje dziewicze może począć i po\petasti{ro}\elaphron{dzić} Sy\oligon{na}?\pauza \petasti{Po}\elaphron{wiedz} \oligon{mi}}. A on w bojaźni i czci rzekł Jej wo\!\oligon{ła}\!jąc:}

\noindent\tertia{\witaj}, wtajemniczona w nie\oligon{wy}\petasti{mo}wną radę,\\
\witaj, pełna wiary w sprawy \kentemata{mil}\petasti{cze}nia godne,\\
\witaj, któraś przedsmakiem cu\oligon{dów} \petasti{Chry}stusowych,\\
\witaj, pełnio wszystkiego, co \oligon{o} \petasti{Nim} jest prawdą.

\vspace{1em}

\noindent\oligon{\witaj}, drabino, po której sam \oligon{Bóg} \petasti{z nie}ba zstąpił,\\
\witaj, moście, wiodący z zie\oligon{mi} \petasti{ku} niebiosom,\\
\witaj, cudzie, o którym słów \oligon{brak} \petasti{jest} aniołom,\\
\witaj, rano bolesna za\oligon{da}\petasti{na} demonom.

\vspace{1em}

\noindent\oligon{\witaj}, Światłość rodząca w spo\oligon{sób} \petasti{nie}wymowny,\\
\witaj, któraś nikomu nie wyjawiła \enquote{Jak} \oligon{swej} \petasti{ta}jemnicy,\\
\witaj, która przekraczasz wie\oligon{dzę} \petasti{wszy}stkich mędrców,\\
\witaj, która wierzącym roz\!\oligon{ja}\!\petasti{śniasz} umysły.

\vspace{1em}

\witajOblubienico