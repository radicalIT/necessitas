\init[-5em]{G}{dy usłyszeli \petasti{pas}\apostrofos{te}rze\pauza aniołów śpiewanie o przyjściu Chrystu\apostrofos[3pt]{sa} \apostrofos[0pt]{w cie}le,\pauza pobiegli jak do Pasterza,\pauza \oligon{uj}rzeli Go jak Baranka bez skazy,\pauza na łonie Ma\petasti{ry}\!\elaphron{i } się \oligon{pasł},\pauza a oni hymn Jej śpie\oligon{wa}li:}

\noindent\tertia{\witaj}, Matko Baran\oligon{ka} \petasti{i} Pasterza,\\
\witaj, zagrodo du\oligon{cho}\petasti{wych} owieczek,\\
\witaj, obrono od wil\oligon{ków} \petasti{nie}widzialnych,\\
\witaj, bramy raju nam \oligon{o}\!\!\petasti{twi}erająca.

\vspace{1em}

\noindent\oligon{\witaj}, bo z ziemią śpiewają ra\oligon{dos}\petasti{ne} niebiosa,\\
\witaj, bo z niebem pląsa \kentemata{szczę}\petasti{śli}wa ziemia,\\
\witaj, Ty, apostołów nie\oligon{mil}\petasti{kną}ce usta,\\
\witaj, niezwyciężone męstwo zwycięstwa \kentemata{wie}\petasti{niec} noszących.

\vspace{1em}

\noindent\oligon{\witaj}, podporo wia\oligon{ry} \petasti{na}szej mocna,\\
\witaj, łaski \kentemata{do}\petasti{wo}dzie jasny,\\
\witaj, która pie\oligon{kło} \petasti{o}\!gałacasz,\\
\witaj, która nas chwa\!\oligon{łą} \petasti{przy}odziewasz.

\vspace{1em}

\noindent\oligon{\witaj}, Oblubienico Dziewicza