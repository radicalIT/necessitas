\small
Celem tego opracowania było przygotowanie zapisu możliwie intuicyjnego. Podczas śpiewu wystarczy zwracać uwagę przede wszystkim na pogrubione sylaby. Dla większej precyzji w notacji wprowadziłem również poniższe oznaczenia, zaczerpnięte z tradycji bizantyjskiej:
\begin{description}
    \item [\comm{Petastē}] \enquote{\scalebox{1.3}{\ByzSym{1D049}}}\hfill\\podniesienie głosu o sekundę (+1) pod akcentem, np.:\\\includegraphics{Notes/Legenda/petaste.png}
    \item [\comm{Oligon}] \enquote{\ByzSym{1D047}}\hfill\\podniesienie głosu o sekundę (+1), np.:\\\includegraphics{Notes/Legenda/oligon.png}
    \item [\comm{Ison z Kentēmatą}] \enquote{\scalebox{1.3}{\stackon[-3.5pt]{\ByzSym{1D046}}{\ByzSym{1D0F0}}}}\hfill\\przeciągnięcie głosu i podniesienie o sekundę (0+1), np.:\\\includegraphics{Notes/Legenda/ison_i_kentemata.png}
    \clearpage
    \item [\comm{Olígon z Kéntēmą}] \enquote{\raisebox{-3pt}{\scalebox{1.3}{\stackon[-2pt]{\hspace{5pt}\ByzSym{1D0F1}}{\ByzSym{1D047}}}}}\hfill\\podniesienie głosu o tercję (+2), np.:\\\includegraphics{Notes/Legenda/oligon_i_kentema.png}
    \item [\comm{Apóstrophos}] \enquote{\scalebox{1.6}{\ByzSym{1D051}}}\hfill\\obniżenie głosu o sekundę (-1), np.:\\\includegraphics{Notes/Legenda/aposrophos.png}
    \item [\comm{Elaphron}] \enquote{\scalebox{1.3}{\raisebox{7pt}{\rotatebox{180}{\ByzSym{1D049}}}}}\hfill\\obniżenie głosu o tercję (-2), np.:\\\includegraphics{Notes/Legenda/elaphron.png}
    \item [\comm{Pauza}] \enquote{\scalebox{1.2}{’}}\hfill\\pauza na oddech, np.:\\\includegraphics{Notes/Legenda/pauza.png}
\end{description}
\normalsize