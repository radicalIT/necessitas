\titleformat{\section}
{\centering\normalfont\Huge\wfont\color{red}}
{\thesection.}{1em}{}

\setstackgap{S}{1pt}

\newfontfamily\bizfont[
  Path=./Fonts/,
  Extension=.ttf
]{Symbola}

\newfontfamily\ortfont[
  Path=./Fonts/,
  Extension=.ttf,
  AutoFakeBold=3.0
]{Abadesa}

\newfontfamily\wfont[
  Path=./Fonts/,
  Extension=.ttf
]{KingthingsL}

\setmainfont{Besley}[
  BoldFont = *-Regular,
  BoldFeatures = {FakeBold=3}
]

\definecolor{beż}{RGB}{255, 250, 230}
\definecolor{darkgray}{gray}{0.25} % 0 = czarny, 1 = biały
\pagecolor{beż}

\AddToShipoutPictureBG{%
  \begin{tikzpicture}[remember picture,overlay]
    \draw[red, line width=0.75pt]
      ([xshift=0.25cm,yshift=-0.25cm]current page.north west) rectangle
      ([xshift=-0.25cm,yshift=0.25cm]current page.south east);
  \end{tikzpicture}%
}

\linespread{1.8}
% \addtolength{\textheight}{20pt}



% Fancyhdr ustawienia
% \pagestyle{fancy}
% \fancyhf{}
% \renewcommand{\headrulewidth}{0pt}

% Nagłówki:
% \fancyhead[RO]{\thepage}             % numer po lewej (prawa strona)
% \fancyhead[CE]{\leftmark}            % chapter wyśrodkowany (prawa strona)
% \fancyhead[LE]{\thepage}             % numer po prawej (lewa strona)
% \fancyhead[CO]{\rightmark}           % section wyśrodkowany (lewa strona)

% \fancyhead[CE]{%
%   \leftmark\\[-3pt]  % tytuł chaptera i przesunięcie w górę (zmień wg potrzeby)
%   \color{red}\rule{0.8\headwidth}{0.5pt} % czerwona linia szerokości 95%
% }

% \fancyhead[CO]{%
%   \rightmark\\[-3pt]  % tytuł chaptera i przesunięcie w górę (zmień wg potrzeby)
%   \color{red}\rule{0.8\headwidth}{0.5pt} % czerwona linia szerokości 95%
% }


\setlength{\baselineskip}{1.2\baselineskip}