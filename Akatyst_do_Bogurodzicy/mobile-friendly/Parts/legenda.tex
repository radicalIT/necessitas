% \small
Celem tego opracowania było przygotowanie zapisu możliwie intuicyjnego. Podczas śpiewu wystarczy zwracać uwagę przede wszystkim na pogrubione sylaby. Dla większej precyzji w notacji wprowadziłem również poniższe oznaczenia, zaczerpnięte z tradycji bizantyjskiej:

\textbf{\comm{Petastē}} \enquote{\scalebox{1.3}{\ByzSym{1D049}}}\\
podniesienie głosu o sekundę (+1) pod akcentem, np.:\\
\includegraphics{Notes/Legenda/petaste.png}

\vfill
\begin{center}
    \makebox[\linewidth]{%
        \includegraphics[width=0.18\textwidth]{Sources/Ryciny/cross2.png}
    }
\end{center}
\clearpage

\textbf{\comm{Oligon}} \enquote{\ByzSym{1D047}}\\
podniesienie głosu o sekundę (+1), np.:\\
\includegraphics{Notes/Legenda/oligon.png}

\textbf{\comm{Ison z Kentēmatą}} \enquote{\scalebox{1.3}{\stackon[-3.5pt]{\ByzSym{1D046}}{\ByzSym{1D0F0}}}}\\
przeciągnięcie głosu i podniesienie o sekundę (0+1), np.:\\
\includegraphics{Notes/Legenda/ison_i_kentemata.png}

\textbf{\comm{Olígon z Kéntēmą}} \enquote{\raisebox{-3pt}{\scalebox{1.3}{\stackon[-2pt]{\hspace{5pt}\ByzSym{1D0F1}}{\ByzSym{1D047}}}}}\\
podniesienie głosu o tercję (+2), np.:\\
\includegraphics{Notes/Legenda/oligon_i_kentema.png}

\vfill
\begin{center}
    \makebox[\linewidth]{%
        \includegraphics[width=1\textwidth]{Sources/Ryciny/Mszał/musica_sacra.png}
    }
\end{center}
\clearpage

\textbf{\comm{Apóstrophos}} \enquote{\scalebox{1.6}{\ByzSym{1D051}}}\\
obniżenie głosu o sekundę (-1), np.:\\
\includegraphics{Notes/Legenda/aposrophos.png}

\textbf{\comm{Elaphron}} \enquote{\scalebox{1.3}{\raisebox{7pt}{\rotatebox{180}{\ByzSym{1D049}}}}}\\
obniżenie głosu o tercję (-2), np.:\\
\includegraphics{Notes/Legenda/elaphron.png}

\textbf{\comm{Pauza}} \enquote{\scalebox{1.2}{’}}\\
pauza na oddech, np.:\\
\includegraphics{Notes/Legenda/pauza.png}
% \normalsize
\vfill
\begin{center}
    \makebox[\linewidth]{%
        \includegraphics[width=0.22\textwidth]{Sources/Ryciny/aniol.png}
    }
\end{center}