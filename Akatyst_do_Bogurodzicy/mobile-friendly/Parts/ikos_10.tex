\init{S}{pieszą do Ciebie dziewice,\pauza boś murem im obronnym \petasti{i}~\apostrofos{wszyst}kim,\pauza co Cię wzywają, Bogurodzico \apostrofos{Dzie}\apostrofos[1pt]{wi}co.\pauza Nieba i ziemi Stworzyciel \oligon{przy}ozdobił Cię łaską, o \petasti{Nie}\elaphron{ska}la\oligon[1pt]{na}.\pauza A zamieszkawszy w Twoim łonie\pauza nauczył wszystkich wznosić głos do \oligon{Cie}bie:}

\sep

\noindent\tertia{\witaj}, o wzniosła ko\oligon{lum}\petasti{no} dziewictwa,\\
\witaj, o bramo wiecz\oligon{ne}\petasti{go} zbawienia,\\
\witaj, pierwszy owocu od\!\oligon{ro}\!\petasti{dze}nia
\nl w Duchu,\\
\witaj, szafarko wszech \oligon{do}\petasti{bro}dziejstw
\nl Bożych.

\clearpage

\noindent\oligon{\witaj}, która odradzasz tych, co
\nl w grze\oligon{chu} \petasti{są} poczęci,\\
\witaj, dająca mądrość tym, \oligon{co}
\nl \petasti{nie}roztropni,\\
\witaj, odpędzająca tego, co \oligon{du}\petasti{sze}
\nl uwodzi,\\
\witaj, rodząca Siew\!\oligon{cę} \petasti{nie}winności.

\sep

\noindent\oligon{\witaj}, alkowo prze\oligon{czys}\petasti{tych} zaślubin,\\
\witaj, zaślubiająca du\oligon{sze} \petasti{wier}ne Panu,\\
\witaj, o piękna dzie\oligon{wic} \petasti{ży}wicielko,\\
\witaj, wesela dusz świę\oligon{tych}
\nl \petasti{przy}jaciółko.

\sep

\witajOblubienico


\vfill
\begin{center}
    \makebox[\linewidth]{%
        \includegraphics[width=0.7\textwidth]{Sources/Ryciny/Mszał/nmp2.png}
    }
\end{center}