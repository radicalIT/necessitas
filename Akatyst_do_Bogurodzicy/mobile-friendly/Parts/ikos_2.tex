\init{C}{zeka Dziewica z u\petasti{prag}\apostrofos{nie}niem\pauza na poznanie niepozna\apostrofos{wal}\apostrofos{ne}go\pauza i rzecze Bożemu słudze:\pauza \enquote{\oligon{Czyż} łono moje dziewicze może począć i po\petasti{ro}\elaphron{dzić} Sy\oligon{na}?\pauza \petasti{Po}\elaphron{wiedz} \oligon{mi}}.\pauza A on w bojaźni i czci rzekł Jej wo\!\oligon{ła}\!jąc:}

\sep

\noindent\tertia{\witaj}, wtajemniczona w nie\oligon{wy}\petasti{mo}wną
\nl radę,\\
\witaj, pełna wiary w sprawy \kentemata{mil}\petasti{cze}nia
\nl godne,\\
\witaj, któraś przedsmakiem cu\oligon{dów}
\nl \petasti{Chry}stusowych,\\
\witaj, pełnio wszystkiego, co \oligon{o} \petasti{Nim}
\nl jest prawdą.

\begin{center}
    \makebox[\linewidth]{%
        \includegraphics[width=0.15\textwidth]{Sources/Ryciny/Mszał/laus_deo.png}
    }
\end{center}

\clearpage

\noindent\oligon{\witaj}, drabino, po której sam \oligon{Bóg}
\nl \petasti{z nie}ba zstąpił,\\
\witaj, moście, wiodący z zie\oligon{mi} \petasti{ku}
\nl niebiosom,\\
\witaj, cudzie, o którym słów \oligon{brak} \petasti{jest}
\nl aniołom,\\
\witaj, rano bolesna za\oligon{da}\petasti{na} demonom.

\sep

\noindent\oligon{\witaj}, Światłość rodząca w spo\oligon{sób}
\nl \petasti{nie}wymowny,\\
\witaj, któraś nikomu nie wyjawiła \enquote{Jak}
\nl \oligon{swej} \petasti{ta}jemnicy,\\
\witaj, która przekraczasz wie\oligon{dzę}
\nl \petasti{wszy}stkich mędrców,\\
\witaj, która wierzącym roz\!\oligon{ja}\!\petasti{śniasz}
\nl umysły.

\sep

\witajOblubienico