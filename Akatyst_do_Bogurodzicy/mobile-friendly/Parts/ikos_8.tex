\init{O}{Słowo, przebywające w pełni do\petasti{cze}\apostrofos{sno}ści,\pauza Ty nie opuszczasz wieczności \apostrofos{swej} \apostrofos{wca}le.\pauza Jesteś wśród nas nieogarnione.\pauza \oligon{Al}bowiem Boskie ku nam zejście\pauza nie było tylko \petasti{zmia}\elaphron{ną} miej\oligon{sca}\pauza te naro\petasti{dzi}\elaphron{ny} z Dziewi\oligon{cy}\pauza -- co Boga pełna słucha naszych \oligon{gło}sów:}

\sep

\noindent\tertia{\witaj}, Ty, co ogarniasz Nie\!\!\oligon[1pt]{o}\!\!\petasti{gar}nionego,\\
\witaj, bramo wznios\oligon{łej} \petasti{ta}jemnicy,\\
\witaj, nowino sprzeciw bu\oligon{dzą}\petasti{ca}
\nl niewiernych,\\
\witaj, chwało nieoba\!\oligon{lo}\!\petasti{na} wierzących.


\vfill
\begin{center}
    \makebox[\linewidth]{%
        \includegraphics[width=0.8\textwidth]{Sources/Ryciny/Mszał/harfa.png}
    }
\end{center}

\clearpage

\sep

\noindent\oligon{\witaj}, tronie najświętszego Tego, co \oligon{jest}
\nl \petasti{nad} Cheruby,\\
\witaj, mieszkanie wzniosłe Tego, co
\nl \oligon{po}\petasti{nad} Serafy,\\
\witaj, która jednoczysz, co
\nl \oligon{nie}\petasti{po}jednane,\\
\witaj, która dziewiczość łą\oligon{czysz}
\nl \petasti{z ma}cierzyństwem.

\sep

\noindent\oligon{\witaj}, przez którą prze\oligon{stęp}\petasti{stwo}
\nl się gładzi,\\
\witaj, przez którą raj \oligon{nam} \petasti{się} otwiera,\\
\witaj, kluczu kró\oligon{les}\petasti{twa} Chrystusa,\\
\witaj, nadziejo szczę\oligon{śli}\petasti{wej} wieczności.

\sep

\witajOblubienico

\clearpage