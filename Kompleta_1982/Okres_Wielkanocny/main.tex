% !TEX program = LuaLaTeX
\documentclass[10pt,a5paper,twoside]{extbook}

\usepackage[osf,p]{libertine}
\usepackage[autocompile]{gregoriotex}

\usepackage[polish]{babel}
\usepackage[T1]{fontenc}
\usepackage[autostyle=true]{csquotes}
\usepackage{lipsum}
\usepackage{xcolor}
\usepackage{geometry}

\usepackage{titlesec}
\usepackage{titletoc}

\usepackage{fontspec}
\setmainfont{Book Antiqua}

\usepackage{etoolbox}
\usepackage{pdfpages}

\usepackage{setspace}

\newcommand{\fakesc}[2]{\MakeUppercase{#1}\scalebox{0.75}{\MakeUppercase{#2}}}

\newcommand{\ikos}[1]{\fakesc{I}{kos} #1}
\newcommand{\kondakion}[1]{\fakesc{K}{ondakion} #1}

\newcommand{\inputIkos}[1]{\input{Parts/ikos_#1.tex}}
\newcommand{\inputKondakion}[1]{\input{Parts/kondakion_#1.tex}}

\newcommand{\ByzSym}[1]{\textcolor{darkgray}{\scalebox{1.3}{\bizfont\char"#1}}}

\newcommand{\init}[2]{
    \lettrine[lraise=0.1, nindent=0em, slope=-.5em]{\redortho{#1}}{#2}
}

\newcommand{\newTitle}[2]{
    \begin{center}
        {
            {\color{red}\fontsize{53}{53}\ortfont #1}\\
            {\Large\wfont #2}
        }
    \end{center}
}

\newcommand{\redortho}[1]{{\textcolor{red}{\ortfont#1}}}

\newcommand{\comm}[1]{\textcolor{red}{#1}}
\newcommand{\sep}{\vspace*{1cm}}
\newcommand{\sepSmall}{\vspace*{0.5cm}}
\newcommand{\sepBig}{\vspace*{1.5cm}}

\newcommand{\nl}{\vspace*{-1mm}\\\hspace*{8mm}}
\newcommand{\cutLine}{\vspace*{-\baselineskip}}

\newcommand{\witaj}{\noindent{\fontsize{17}{17}\comm{\wfont W}}itaj}

\newcommand{\allelujaiii}{
    \tertia{\fontsize{17}{17}\comm{\wfont A}l}leluja, \comm{\Large\wfont a}lle{\normalsize\petasti{lu}ja}, \comm{\Large\wfont a}lleluja.
}
\newcommand{\alleluja}{
    \apostrofos{\fontsize{13}{13}\comm{\wfont A}l}\!\oligon{le}\!\!\oligon[3pt]{lu}\!ja.
}
\newcommand{\witajOblubienico}{\noindent\oligon{\witaj}, Oblubie\petasti{ni}co Dziewicza.}

\newcommand{\V}[1]{\textcolor{red}{$\mathbf{\not{\mkern -3mu \mathrm{V}}}.$}~#1}
\newcommand{\R}[1]{\textcolor{red}{$\mathbf{\not{\mkern -3mu \mathrm{R}}}.$}~#1}

% Znaki bizantyjskie
\newcommand{\ison}[1]{\stackon[2pt]{#1}{\ByzSym{1D046}}}
\newcommand{\oligon}[2][0pt]{\stackon[#1]{#2}{\vspace{#1}\ByzSym{1D047}}}
\newcommand{\apostrofos}[2][2pt]{\stackon[#1]{#2}{\ByzSym{1D051}}}
\newcommand{\tertia}[1]{\stackon[-1pt]{#1}{\stackon[-2pt]{\hspace{5pt}\ByzSym{1D0F1}}{\ByzSym{1D047}}}}
\newcommand{\elaphron}[1]{\stackon[4pt]{#1}{\rotatebox{180}{\ByzSym{1D049}}}}
\newcommand{\petasti}[1]{\stackon[2pt]{{\bf#1}}{\ByzSym{1D049}}}
\newcommand{\kentemata}[1]{\stackon[0pt]{#1}{\stackon[-4pt]{\ByzSym{1D046}}{\ByzSym{1D0F0}}}}
\newcommand{\pauza}{\textcolor{darkgray}{\raisebox{2pt}{\ \scalebox{1.5}{’}\ }}}
\titleformat{\chapter}
{\normalfont\Large\bfseries}{\thechapter.}{1em}{}

\titleformat{\section}
{\center\normalfont\large\bfseries}{\thesection.}{1em}{}

\titleformat{\subsection}
{\center\normalfont\normalsize\bfseries}{\thesubsection.}{1em}{}
\titleformat{\subsubsection}
{\normalfont\small\bfseries}{\thesubsection.}{1em}{}

\pagestyle{plain}

% \titleformat{\chapter}[display]
%   {\normalsize \huge  \color{black}}%
%   {\flushright\normalsize \color{RoyalRed}%
%    \MakeUppercase{\chaptertitlename}\hspace{1ex}%
%    {\fontfamily{mdugm}\fontsize{60}{60}\selectfont\thechapter}}%
%   {10 pt}%
%   {\bfseries\huge}%
\makeatletter
\def\@makechapterhead#1{%
  {\parindent \z@ \raggedright \normalfont
      % \ifnum \c@secnumdepth >\m@ne
      %     \Large\bfseries\thechapter.\space
      % \fi
      \interlinepenalty\@M
      \huge \bfseries #1\par\nobreak
      \vskip 40\p@
    }}
\makeatother

\setcounter{secnumdepth}{0}

\renewcommand{\cleardoublepage}{\clearpage}
% \renewcommand{\thesection}{abc}
% \let\cleardoublepage=\clearpage

\newgeometry{
  top=20mm,
  bottom=20mm,
  outer=12mm,
  inner=15mm,
}

\grechangestaffsize{15}
\gresetinitiallines{0}
\grechangestyle{annotation}{\small\bfseries}
\grechangestyle{initial}{\fontsize{32}{35}\mdseries}
\grechangedim{spaceabovelines}{2mm}{scalable}
\gresetheadercapture{sigla}{grecommentary[2mm]}{}


% \titleformat{\subsection}
%   {\normalfont\normalsize\itshape}{\thesubsection.}{1em}{}

% \titleformat{\subsubsection}
%   {\normalfont\normalsize\itshape}{\thesubsubsection.}{1em}{}

\nonstopmode

% TO DO:
% + Skrócić zapis responsorium
% + Przeglądnąć rozciągnięcia
% Poprawić rozciągnięcia w oracji do Savle
% + Poprawić alleluje - 4x w oktawie, 3x normalnie
% + Wywalić antyfony bez alleluji
% + Dodać 2 wersje Regina caeli
% + Powywalać pauzy w Haec Dies zgodnie z nieszporami
% + przecinek po emmanuel, średnia kreska i alleluia

\title{Kompleta}
% Na podstawie Proprium Officiorum 1982r.
% \author{br. Kacper Osika OP}

\begin{document}
\includepdf[pages={1}]{Materiały/cover_page.pdf}

\thispagestyle{empty}
\color{white}\tiny
-
\vfill
\begin{center}
    \includegraphics[width=175px]{Ryciny/Mszał/agnus_dei2.jpg}
\end{center}
\vfill
\color{black}\normalsize
\clearpage

\chapter{Modlitwy stałe}
\section*{Modlitwa przed oficjum}
\parbox{12cm}{
    \setstretch{1.25}
    \comm{Tę modlitwę można odmówić indywidualnie według uznania:}\\
    Panie Jezu, w łączności z tą boską intencją, w której sam, będąc tu na ziemi, składałeś chwałę Bogu, ofiaruję Ci to nabożeństwo.}
\section*{Rozpoczęcie}
% \V{Pobłogosław mnie, Ojcze}
\gregorioscore{GABC/jube.gabc}
% \R{Noc spokojną i śmierć szczęśliwą niech nam da Bóg wszechmogący, Ojciec i Syn, i Duch Święty.}
% \V{Amen.}
\gregorioscore{GABC/noctem.gabc}
\section*{Krótka lekcja (1 P 5, 8-9)}
% Bądźcie trzeźwi, czuwajcie. Przeciwnik wasz, diabeł, jak lew ryczący krąży szukając kogo pożreć. Mocni w wierze przeciwstawcie się jemu.
\gregorioscore{GABC/lector.gabc}
% \R{Ty zaś, Panie, zmiłuj się nad nami.}
% \V{Bogu niech będą dzięki.}
\gregorioscore{GABC/tu_autem.gabc}
% \R{Wspomożenie nasze w imieniu Pana}
% \V{Który stworzył niebo i ziemię}
\gregorioscore{GABC/adjutorium.gabc}\\

\begin{center}
    \parbox{10cm}{
        \comm{Prowadzący zaczyna:}\\
        Ojcze nasz...\\
        \comm{Wszyscy odmawiają po cichu.}
        \section*{Akt pokuty}
        \R{Spowiadam się Bogu wszechmogącemu,\\
            Najświętszej Maryi zawsze Dziewicy,\\
            świętemu naszemu Ojcu Dominikowi\\
            i wszystkim Świętym, i wam Bracia:\\
            że bardzo zgrzeszyłem,\\
            myślą, mową, uczynkiem i zaniedbaniem:\\
            \comm{Bijąc się w piersi:}\\
            moja wina, moja wina, moja bardzo wielka wina.\\
            \comm{Mówiąc dalej:}\\
            Proszę was o modlitwę za mną.}\\
        \V{Niech się zmiłuje nad tobą Bóg wszechmogący, a odpuściwszy ci grzechy, doprowadzi cię do życia wiecznego.}\\
        \R{Amen.}\\
    }
\end{center}
\begin{center}
    \parbox{10cm}{
        \V{Spowiadam się Bogu wszechmogącemu,\\
            Najświętszej Maryi zawsze Dziewicy,\\
            świętemu naszemu Ojcu Dominikowi\\
            i wszystkim Świętym, i tobie Ojcze:\\
            że bardzo zgrzeszyłem,\\
            myślą, mową, uczynkiem i zaniedbaniem:\\
            \comm{Bijąc się w piersi:}\\
            moja wina, moja wina, moja bardzo wielka wina.\\
            \comm{Mówiąc dalej:}\\
            Proszę cię o modlitwę za mną.}\\
        \R{Niech się zmiłuje nad wami Bóg wszechmogący, a odpuściwszy wam grzechy, doprowadzi was do życia wiecznego.}\\
        \V{Amen.}\\
    }
\end{center}

\comm{Czyniąc znak krzyża św. na czole, ustach i piersi:} % \R{Nawróć nas Boże, nasz zbawco.}
\gregorioscore{GABC/converte_nos_1.gabc}\\
\comm{Czyniąc podobnie:}
% \V{I odwróć swój gniew od nas.}
\gregorioscore{GABC/converte_nos_2.gabc}\\
\comm{Czyniąc normalny znak krzyża:}
% \R{Boże, wejrzyj ku wspomożeniu memu.}
\gregorioscore{GABC/deus_in_adjutorium_1.gabc}\\
\clearpage
\comm{Czyniąc podobnie:}
% \V{Panie, pośpiesz ku ratunkowi memu.
% Chwała Ojcu i Synowi, * i Duchowi Świętemu.\\
% Jak była na początku, teraz i zawsze, * i na wieki wieków. Amen. Alleluja.}
\gregorioscore{GABC/deus_in_adjutorium_2.gabc}\\
\comm{Od siedemdziesiątnicy do wielkanocy zamiast Alleluja śpiewa się poniższą frazę:}
% Chwała Tobie Panie, Królu wiecznej chwały
\gregorioscore{GABC/laus_tibi.gabc}
\clearpage
\section*{Hymn}
\gresetinitiallines{1}

% Christe, qui lux es et dies,
% noctis tenébras détegis,
% lucisque lumen créderis,
% lumen beatum pnédicans.

% Precamur sancte Domine,
% defénde nos in hac nocte,
% sit nobis in te réquies,
% quiétam noctem tribue.

% Ne gravis somnus irruat,
% nec hostis nos subripiat,
% nec caro illi conséntiens
% nos tibi reos statuat.

% Oculi somnum capiant,
% cor ad te semper vigilet,
% déxtera tua protegat
% famulos, qui te diligunt.

% Defénsor noster aspice,
% insidiantes réprime,
% gubérna tuos famulos,
% quos sanguine mercatus es.

% Meménto nostri, Domine,
% in gravi isto corpore;
% qui es defénsor animre,
% adésto nobis, Domine.

% Présta, Pater omnipotens,
% per Iesum Christum Dominum,
% qui tecum in perpétuum
% regnat cum Sancto Spiritu.

% Chryste, Tyś Światłem jest i Dniem,
% Który rozprasza nocy cień.
% Tyś ze Światłości światła blask,
% Jasności świętej głosisz dzień.

% Panie, do Twych pukamy bram,
% Niech nas obroni Twoja moc.
% Naszym pokojem bądź Ty sam
% I daj spokojną, cichą noc.

% Spraw, by nas nie zmógł ciężar snu
% I mie oszukał wrogi syn,
% Niech ciało nie ulegnie mu,
% Nas obarczając jarzmem win.

% Choć na powieki spada sen,
% Lecz serce czuwa w dzień i w noc.
% Czeladkę, która kocha Cię
% Niech broni Twej prawicy moc.
\gregorioscore{GABC/Hymny/christe_qui_lux_1.gabc}\vspace{3mm}
\gresetinitiallines{0}
\clearpage\noindent
\begin{center}
    \parbox{10cm}{
        \comm{W trakcie śpiewania wersu \enquote{Których nabyłeś własną Krwią} klęka się dla uczczenia Męki Pańskiej.}
    }
\end{center}

% Obróć, Obrońco, na nas wzrok,
% Stłum czyhających zgraję złą,
% Sług swoich rozrządź każdy krok,
% Których nabyłeś własną Krwią.

% Pamiętaj o nas, Panie nasz,
% Gdy nas przygniata ciała znój.
% Obrońcą naszym jesteś Ty,
% Więc ku pomocy zawsze stój.

% Ojcze Wszechmocny, prosim Cię,
% W Imię Jezusa usłysz nas,
% Co z Tobą poprzez wieków bieg
% Króluje z Duchem Świętym wraz. Amen.
\gregorioscore{GABC/Hymny/christe_qui_lux_2.gabc}\vspace{3mm}\noindent

\vfill
\begin{center}
    \includegraphics[width=160px]{Ryciny/Mszał/baner5.jpg}
\end{center}
\clearpage
\begin{center}
    \parbox{10cm}{
        \comm{W uroczystości i święta maryjne można zastosować poniższe zakończenie hymnu:}\\
    }
\end{center}
\gregorioscore{GABC/Hymny/maria_mater_gratiae.gabc}\vspace{3mm}\noindent

\vfill
\begin{center}
    \includegraphics[width=320px]{Ryciny/Mszał/nmp_nieustajacej_pomocy2.jpg}
\end{center}
\vfill
\clearpage

\thispagestyle{empty}
\color{white}\tiny
-
\vfill
\begin{center}
    \includegraphics[width=140px]{Ryciny/Mszał/blogoslawienstwo_greckie.jpg}
\end{center}
\vfill
\color{black}\normalsize
\clearpage

\section*{Psalmy i czytania}
\begin{center}
    \parbox{10cm}{
        \comm{Z odpowiedniego dnia: s. \pageref{ch:psalterz}.}\\
    }
\end{center}
\section*{Responsorium}
% \R{W ręce Twoje, Panie, * Powierzam ducha mojego.}
% \V{W ręce Twoje, Panie, / Powierzam ducha mojego.}
% \R{Ty nas odkupiłeś, Panie, Boże wierny.}
% \V{Powierzam ducha mojego.}
% \R{Chwała Ojcu i Synowi, i Duchowi Świętemu.}
% \V{W ręce Twoje, Panie, / Powierzam ducha mojego.}
\gregorioscore{GABC/in_manus.gabc}\\
% \comm{W okresie Męki Pańskiej pomija się Chwała Ojcu.}\\\\
\begin{center}
    \parbox{10cm}{
        \comm{W uroczystości można dodać podwójne "Alleuja":}
    }
\end{center}
\gregorioscore{GABC/in_manus_alleluja.gabc}
\vfill
\begin{center}
    \includegraphics[width=185px]{Ryciny/Mszał/harfa.jpg}
\end{center}
\vfill
\clearpage
\comm{Ant.} Strzeż nas, Panie, gdy czuwamy, \comm{*} podczas snu nas osłaniaj, \comm{/} abyśmy czuwali z Chrystusem \comm{/} i odpoczywali w pokoju.

\section*{Kantyk Symeona}\\
\begin{tabbing}
    \hspace{4mm} \= \kill
    Teraz, o Panie, pozwól odejść swemu słudze w pokoju, *\\
    według słowa Twojego,\\
    \>Bo moje oczy ujrzały Twoje zbawienie, *\\
    \>któreś przygotował wobec wszystkich narodów:\\
    Światło na oświecenie pogan *\\
    i chwałę ludu Twego, Izraela.
\end{tabbing}

\comm{Ant.} Strzeż nas, Panie, gdy czuwamy, \comm{/} podczas snu nas osłaniaj, \comm{/} abyśmy czuwali z Chrystusem \comm{/} i odpoczywali w pokoju.

\vfill
\begin{center}
    \includegraphics[width=340px]{Ryciny/Herby/97.jpg}
\end{center}
\vfill
\clearpage
\section*{Oracja końcowa}
\gresetinitiallines{0}
\V{Módlmy się.\\Boże, nasz Ojcze, nawiedź ten dom i oddal od niego wszelkie wrogie zasadzki; \comm{†} niech w nim przebywają Twoi święci aniołowie i strzegą nas w pokoju, a Twoje błogosławieństwo niech nam zawsze towarzyszy. Przez Chrystusa Pana naszego. }
\R{Amen.}

\section*{Zakończenie modlitwy}
\V{Niech Bóg wszechmogący i miłosierny da nam noc spokojną i koniec szczęśliwy.}
\R{Amen.}

\comm{Na zakończenie odśpiewuje się jedną z antyfon -- s. \pageref{ch:antyfony}.}

\vfill
\begin{center}
    \includegraphics[width=340px]{Ryciny/Herby/7.jpg}
\end{center}
\vfill
\clearpage
\section*{Modlitwa po komplecie}
\setstretch{1.25}
\comm{Tę modlitwę można odmówić według uznania:}\\
Wieczna chwała, cześć, moc i sława niech będzie od wszelkiego stworzenia Przenajświętszej i nierozdzielnej Trójcy, człowieczeństwu Pana naszego Jezusa Chrystusa ukrzyżowanego, jako też płodnemu panieństwu błogosławionej Maryi, zawsze Dziewicy, oraz wszystkim Świętym, nam zaś niechaj Pan udzieli odpuszczenia wszystkich grzechów naszych, przez nieskończone wieki wieków. Amen.\\ \\
\V{Błogosławiony żywot Panny Maryi, który nosił Syna Ojca Przedwiecznego.}
\R{Błogosławione piersi, które karmiły Chrystusa Pana.}

\comm{W intecji Ojca Świętego i Kościoła:}\\
Ojcze Nasz...\\
Zdrowaś Maryjo...\\
% \vfill
% \begin{center}
%     \includegraphics[width=50px]{Ryciny/Mszał/blogoslawienstwo_greckie.gif}
% \end{center}
% \vfill
\chapter{Psałterz}
\label{ch:psalterz}
\gresetinitiallines{1}
\section{Po I nieszporach niedziel i uroczystości}
\comm{Ant.} Alleluja, \comm{*} alleluja, alleluja.
% \gregorioscore{GABC/Antyfony/sb_1_z_inc.gabc}
\subsection*{Psalm 4}
\begin{tabbing}
    \hspace{4mm} \= \>\kill
    Kiedy Cię wzywam, odpowiedz mi, Boże,~*\\
    który wymierzasz mi sprawiedliwość.\\
    \>Tyś mnie wydźwignął z utrapienia,~*\\
    \>zmiłuj się nade mną i wysłuchaj moją modlitwę.\\
    Jak długo będą ociężałe wasze serca, mężowie?~*\\
    Czemu kochacie marność i szukacie kłamstwa?\\
    \>Wiedzcie, że godnym podziwu czyni Pan swego\\wiernego,~*\\
    \>Pan mnie wysłucha, gdy będę Go wzywał.\\
    Zadrżyjcie i już nie grzeszcie,~*\\
    rozważcie na swych łożach i zamilknijcie.\\
    \>Złóżcie należne ofiary~*\\
    \>i miejcie nadzieję w Panu.
\end{tabbing}
\clearpage
\begin{tabbing}
    \hspace{4mm} \= \>\kill
    Wielu powiada: "Któż nam szczęście ukaże?"~*\\
    Wznieś ponad nami, Panie, światłość Twojego oblicza!\\
    \>Więcej wlałeś radości w moje serce~*\\
    \>niż w czasie obfitych plonów pszenicy i wina.\\
    Spokojnie zasypiam, kiedy się położę, †\\
    bo tylko Ty jeden, Panie,~*\\
    pozwalasz mi żyć bezpiecznie.
\end{tabbing}
\subsection*{Psalm 134}
\begin{tabbing}
    \hspace{4mm} \= \>\kill
    Błogosławcie Pana, wszyscy słudzy Pańscy,~*\\
    którzy przebywacie nocą w Jego domu.\\
    \>Wznieście wasze ręce ku Miejscu Świętemu~*\\
    \>i błogosławcie Pana.\\
    Niech Cię Pan błogosławi ze Syjonu,~*\\
    Ten, który stworzył niebo i ziemię.
\end{tabbing}
% \gregorioscore{GABC/Antyfony/sb_2.gabc}
\comm{Ant.} Alleluja, alleluja, alleluja.

\vfill
\begin{center}
    \includegraphics[width=220px]{Ryciny/Mszał/aniolek.gif.jpg}
\end{center}
\vfill

\clearpage

\subsection*{Czytanie}
\subsubsection*{Pwt 6, 4-7}
Słuchaj, Izraelu: Pan jest naszym Bogiem, Panem jedynym. Będziesz miłował Pana, Boga twojego, z całego swego serca, z całej duszy swojej, ze wszystkich sił swoich. Niech pozostaną w twym sercu te słowa, które ja ci dziś nakazuję. Wpoisz je twoim synom, będziesz o nich mówił przebywając w domu, w czasie podróży, kładąc się spać i wstając ze snu.
% \gregorioscore{GABC/Czytania/sb.gabc}
\vfill
\begin{center}
    \includegraphics[width=230px]{Ryciny/Mszał/jerzy.jpg}
\end{center}
\vfill


\\
\clearpage
\section{Po II nieszporach niedziel i uroczystości}

% \gregorioscore{GABC/Antyfony/nd_z_inc.gabc}
\comm{Ant.} Pan cię okryje swoimi piórami, \comm{*} nie ulękniesz się strachu nocnego.
\subsection*{Psalm 90}
\begin{tabbing}
    \hspace{4mm} \= \>\kill
    Kto się w opiekę oddał Najwyższemu~*\\
    i w cieniu Wszechmocnego mieszka,\\
    \>Mówi do Pana: "Tyś moją ucieczką i twierdzą,~*\\
    \>Boże mój, któremu ufam".\\
    Bo On sam cię wyzwoli z sideł myśliwego~*\\
    i od słowa niosącego zgubę.\\
    \>Okryje cię swoimi piórami, †\\
    \>pod Jego skrzydła się schronisz;~*\\
    \>wierność Jego jest puklerzem i tarczą.\\
    Nie ulękniesz się strachu nocnego~*\\
    ani strzały za dnia lecącej,\\
    \>Ani zarazy skradającej się w mroku,~*\\
    \>ani moru niszczącego w południe.\\
    A choćby tysiąc padło u boku twego †\\
    i dziesięć tysięcy po twojej prawicy,~*\\
    ciebie to nie spotka.~*\\
    \>Ty zaś ujrzysz własnymi oczyma~*\\
    \>zapłatę daną grzesznikom.\\
    Bo Pan jest twoją ucieczką,~*\\
    za obrońcę wziąłeś Najwyższego.\\
    \>Nie przystąpi do ciebie niedola,~*\\
    \>a cios nie dosięgnie twojego namiotu,\\
    Bo rozkazał swoim aniołom,~*\\
    aby cię strzegli na wszystkich twych drogach.\\
    \>Będą cię nosili na rękach,~*\\
    \>abyś stopy nie uraził o kamień.\\
    Będziesz stąpał po wężach i żmijach,~*\\
    a lwa i smoka podepczesz.\\
    \>"Ja go wybawię, bo przylgnął do Mnie,~*\\
    \>osłonię go, bo poznał moje imię.\\
    Będzie Mnie wzywał, a Ja go wysłucham †\\
    i będę z nim w utrapieniu,~*\\
    wyzwolę go i sławą obdarzę.\\
    \>Nasycę go długim życiem~*\\
    \>i ukażę mu moje zbawienie".
\end{tabbing}

% \gregorioscore{GABC/Antyfony/nd.gabc}
\comm{Ant.} Pan cię okryje swoimi piórami, \comm{*} nie ulękniesz się strachu nocnego.

% \vfill
% \begin{center}
%     \includegraphics[width=170px]{Ryciny/Mszał/baner_rex_gloria_christi.png}
% \end{center}
\clearpage

\subsection*{Czytanie}
\subsubsection*{Ap 22, 4-5}
% \gregorioscore{GABC/Czytania/nd.gabc}
Słudzy Pana będą oglądać Jego oblicze, a imię Jego na ich czołach. I odtąd już nocy nie będzie. A nie potrzeba im światła lampy i światła słońca, bo Pan Bóg będzie świecił nad nimi i będą królować na wieki wieków.

\vfill
\begin{center}
    \includegraphics[width=185px]{Ryciny/Mszał/dominicane.jpg}
\end{center}
\vfill\\
\clearpage
\section{Poniedziałek}

\gregorioscore{GABC/Antyfony/pn_z_inc.gabc}
% \comm{w Okresie Wielkanocnym:}
% \gregorioscore{GABC/Antyfony/alleluja.gabc}
\subsection*{Psalm 85}
\begin{center}
    \parbox{8cm}{
        \begin{tabbing}
            \hspace{4mm} \= \>\kill
\color{gray}
Nakłoń swego ucha i wysłuchaj mnie, Panie,~*\\

bo biedny jestem i ubogi.\\
\>Strzeż mojej duszy, bo jestem pobożny,~*\\
\>zbaw sługę Twego, który ufa Tobie.\\
Ty jesteś moim Bogiem, †\\
Panie, zmiłuj się nade mną,~*\\
bo nieustannie wołam do Ciebie.\\
\>Uraduj duszę swego sługi,~*\\
\>ku Tobie, Panie, wznoszę moją duszę.\\
Tyś bowiem, Panie, dobry i łaskawy,~*\\
pełen łaski dla wszystkich, którzy Cię wzywają.\\
\>Wysłuchaj, Panie, modlitwę moją~*\\
\>i zważ na głos mojej prośby.\\
W dniu utrapienia wołam do Ciebie,~*\\
ponieważ Ty mnie wysłuchasz.\\
\>Nie ma wśród bogów równego Tobie, Panie,~*\\
\>ani Twemu dziełu inne nie dorówna.\\
\end{tabbing}}
\end{center}
\begin{center}
\parbox{8cm}{
\begin{tabbing}
\hspace{4mm} \= \>\kill
Przyjdą wszystkie ludy przez Ciebie stworzone †\\
i Tobie, Panie, oddadzą pokłon,~*\\
będą sławiły Twe imię.\\
\>Bo Ty jesteś wielki i czynisz cuda,~*\\
\>tylko Ty jesteś Bogiem.\\
Naucz mnie, Panie, Twej drogi, †\\
bym postępował według Twojej prawdy,~*\\
nakłoń me serce, by lękało się Twego imienia.\\
\>Będę Cię wielbił z całego serca, Panie mój i Boże,~*\\
\>i na wieki będę sławił Twoje imię.\\
Bo wielkie było dla mnie Twoje miłosierdzie,~*\\
a życie moje wyrwałeś z głębin Otchłani.\\
\>Boże, pyszni przeciw mnie powstali †\\
\>i zgraja zuchwalców czyha na me życie,~*\\
\>a nie mają oni względu na Ciebie.\\
Ale Tyś, Panie, Bogiem łaski i miłosierdzia,~*\\
do gniewu nieskory, łagodny i bardzo wierny.\\
\>Wejrzyj na mnie i zmiłuj się nade mną, †\\
\>siły swej udziel słudze Twojemu,~*\\
\>ocal syna swojej służebnicy!\\
Daj mi znak dobroci, †\\
aby ujrzeli ze wstydem ci, którzy mnie nienawidzą,~*\\
żeś Ty, Panie, mnie wspomógł i pocieszył.\\
\>Chwała Ojcu i Synowi,~*\\
\>i Duchowi Świętemu.\\
Jak była na początku, teraz i zawsze,~*\\
i na wieki wieków. Amen.
\end{tabbing}}
\end{center}

\gregorioscore{GABC/Antyfony/pn.gabc}

\vfill

\subsection*{Czytanie}
\gregorioscore{GABC/Czytania/pn.gabc}
\vfill
\begin{center}
    \includegraphics[width=130px]{Ryciny/Mszał/wojciech.jpg}
\end{center}
\vfill\\
\clearpage
\section{Wtorek}

% \gregorioscore{GABC/Antyfony/wt_z_inc.gabc}
\comm{Ant.} Panie, nie ukrywaj przede mną swojego oblicza, \comm{*} bo w Tobie pokładam nadzieję.
\subsection*{Psalm 143}
        \begin{tabbing}
            \hspace{4mm} \= \>\kill
            Usłysz, Panie, modlitwę moją, †\\
            w swojej wierności przyjm moje błaganie,~*\\
            wysłuchaj mnie w swej sprawiedliwości.\\
            \>Nie wzywaj na sąd swojego sługi,~*\\
            \>bo nikt z żyjących nie jest sprawiedliwy przed Tobą.\\
            Albowiem prześladuje mnie nieprzyjaciel, †\\
            moje życie na ziemię powalił,~*\\
            pogrążył mnie w ciemnościach, jak dawno umarłych.\\
            \>A we mnie duch mój omdlewa,~*\\
            \>zamiera we mnie serce.\\
            Wspominam dni dawno minione, †\\
            rozmyślam o wszystkich Twych dziełach,~*\\
            rozważam dzieło rąk Twoich.\\
            \>Wyciągam do Ciebie ręce,~*\\
            \>jak zeschła ziemia pragnie Cię moja dusza.\\\\
            Wysłuchaj mnie prędko, Panie,~*\\
            albowiem duch mój omdlewa.\\
            \>Nie ukrywaj przede mną swojego oblicza,~*\\
            \>bym się nie stał podobny do tych, którzy schodzą\\do grobu.\\
            Daj mi już o świcie doznać łaski Twojej,~*\\
            bo w Tobie pokładam nadzieję.\\
            \>Pokaż mi drogę, którą mam kroczyć,~*\\
            \>bo wznoszę ku Tobie moją duszę.\\
            Wybaw mnie, Panie, od moich wrogów,~*\\
            uciekam się do Ciebie.\\
            \>Naucz mnie spełniać Twoją wolę,~*\\
            \>bo Ty jesteś moim Bogiem.\\
            Niech mnie Twój dobry duch prowadzi~*\\
            po bezpiecznej równinie.\\
            \>Przez wzgląd na Twoje imię, Panie, zachowaj mnie\\przy życiu,~*\\
            \>w swej sprawiedliwości wyprowadź mnie z niedoli.\\
            Chwała Ojcu i Synowi,~*\\
            i Duchowi Świętemu.\\
            \>Jak była na początku, teraz i zawsze,~*\\
            \>i na wieki wieków. Amen.
        \end{tabbing}
% \gregorioscore{GABC/Antyfony/wt.gabc}
\comm{Ant.} Panie, nie ukrywaj przede mną swojego oblicza,~\comm{/} bo w Tobie pokładam nadzieję.

\subsection*{Czytanie}
\subsubsection*{1 P 5, 8-9}
Bądźcie trzeźwi, czuwajcie. Przeciwnik wasz, diabeł, jak lew ryczący krąży szukając kogo pożreć. Mocni w wierze przeciwstawcie się jemu.
% \gregorioscore{GABC/Czytania/wt.gabc}
\vfill
\begin{center}
    \includegraphics[width=200px]{Ryciny/Mszał/aniolek5.jpg}
\end{center}
\vfill\\
\clearpage
\section*{Środa}

\gregorioscore{GABC/Antyfony/sr_1_z_inc.gabc}
\subsection*{Psalm 31}
\begin{center}
    \parbox{8cm}{
        \begin{tabbing}
            \hspace{4mm} \= \>\kill
            \color{gray}
            Panie, do Ciebie się uciekam, †\\
            \color{gray}
            niech nigdy nie doznam zawodu,~*\\
            wybaw mnie w sprawiedliwości Twojej.\\
            \>Nakłoń ku mnie Twego ucha,~*\\
            \>pośpiesz, aby mnie ocalić!\\
            Bądź dla mnie skałą schronienia,~*\\
            warownią, która ocala.\\
            \>Ty bowiem jesteś moją skałą i twierdzą,~*\\
            \>kieruj mną i prowadź przez wzgląd na swe imię.\\
            Wydobądź z sieci zastawionej na mnie,~*\\
            bo Ty jesteś moją ucieczką.\\
            \>W ręce Twoje powierzam ducha mego,~*\\
            \>Ty mnie odkupisz, Panie, wierny Boże.\\
            Chwała Ojcu i Synowi,~*\\
            i Duchowi Świętemu.\\
            \>Jak była na początku, teraz i zawsze,~*\\
            \>i na wieki wieków. Amen.
        \end{tabbing}}
\end{center}
\gregorioscore{GABC/Antyfony/sr_1.gabc}

\gregorioscore{GABC/Antyfony/sr_2_z_inc.gabc}
\subsection*{Psalm 130}
\begin{center}
    \parbox{8cm}{
        \begin{tabbing}
            \hspace{4mm} \= \>\kill
            \color{gray}
            Z głębokości wołam do Ciebie, Panie,~*\\
            \color{gray}
            \comm{†} Panie, wysłuchaj głosu mego.\\
            \>\color{gray}Nachyl swe ucho~*\\
            \>na głos mojego błagania.\\
            Jeśli zachowasz pamięć o grzechach, Panie,~*\\
            Panie, któż się ostoi?\\
            \>Ale Ty udzielasz przebaczenia,~*\\
            \>aby Ci ze czcią służono.\\
            Pokładam nadzieję w Panu, †\\
            dusza moja pokłada nadzieję w Jego słowie,~*\\
            dusza moja oczekuje Pana.\\
            \>Bardziej niż strażnicy poranka~*\\
            \>niech Izrael wygląda Pana.\\
            U Pana jest bowiem łaska,~*\\
            u Niego obfite odkupienie.\\
            \>On odkupi Izraela~*\\
            \>ze wszystkich jego grzechów.\\
            Chwała Ojcu i Synowi,~*\\
            i Duchowi Świętemu.\\
            \>Jak była na początku, teraz i zawsze,~*\\
            \>i na wieki wieków. Amen.
        \end{tabbing}}
\end{center}
\gregorioscore{GABC/Antyfony/sr_2.gabc}

\section*{Czytanie}
\gregorioscore{GABC/Czytania/sr.gabc}

\vfill
\begin{center}
    \includegraphics[width=180px]{Ryciny/Mszał/krzyz14.jpg}
\end{center}
\vfill\\
\clearpage

\section{Czwartek}

\subsection*{Przed Uroczystością Objawienia Pańskiego}
\comm{Ant.} Narodził się nam dziś Zabwiciel, \comm{*} którym jest Chrystus Pan, w mieście Dawidowym.

\subsection*{Od Uroczystości Objawienia Pańskiego}
\comm{Ant.} Światłość ze światłości, \comm{*} objawiłeś się Chryste, któremu Mędrcy dary składają, \comm{/} alleluja, alleluja, \mbox{alleluja}.

\subsection*{Psalm 16}
\begin{tabbing}
\hspace{4mm} \= \>\kill
    Zachowaj mnie, Boże, bo chronię się do Ciebie, †\\
    mówię do Pana: "Tyś jest Panem moim,~*\\
    poza Tobą nie ma dla mnie dobra".\\
    \>Wzbudził On we mnie miłość przedziwną~*\\
    \>do świętych, którzy mieszkają na Jego ziemi.\\
    A wszyscy, którzy idą za obcymi bogami,~*\\
    pomnażają swoje udręki.\\
    \>Nie będę wylewał krwi w ofiarach dla nich,~*\\
    \>nie wymówią ich imion moje wargi.
\end{tabbing}
\clearpage
\begin{tabbing}
    \hspace{4mm} \= \>\kill
    Pan moim dziedzictwem i przeznaczeniem,~*\\
    to On mój los zabezpiecza.\\
    \>Sznur mierniczy szczodrze mi dział wyznaczył,~*\\
    \>jak miłe jest dla mnie dziedzictwo moje!\\
    Błogosławię Pana, który dał mi rozsądek,~*\\
    bo serce napomina mnie nawet nocą.\\
    \>Zawsze stawiam sobie Pana przed oczy,~*\\
    \>On jest po mojej prawicy, nic mną nie zachwieje.\\
    Dlatego cieszy się moje serce i dusza raduje,~*\\
    a ciało moje będzie spoczywać bezpiecznie,\\
    \>Bo w kraju zmarłych duszy mej nie zostawisz~*\\
    \>i nie dopuścisz, bym pozostał w grobie.\\
    Ty ścieżkę życia mi ukażesz, †\\
    pełnię radości przy Tobie~*\\
    i wieczne szczęście po Twojej prawicy.\\
    \>Chwała Ojcu i Synowi,~*\\
    \>i Duchowi Świętemu.\\
    Jak była na początku, teraz i zawsze,~*\\
    i na wieki wieków. Amen.
\end{tabbing}

\subsection*{Przed Uroczystością Objawienia Pańskiego}
\comm{Ant.} Narodził się nam dziś Zabwiciel, \comm{/} którym jest Chrystus Pan, w mieście Dawidowym.

\subsection*{Od Uroczystości Objawienia Pańskiego}
\comm{Ant.} Światłość ze światłości, \comm{/} objawiłeś się Chryste, któremu Mędrcy dary składają, \comm{/} alleluja, alleluja, \mbox{alleluja}.

\subsection*{Czytanie}
\subsubsection*{1 Tes 5, 23}
% \gregorioscore{GABC/Czytania/czw.gabc}
Sam Bóg pokoju niech was całkowicie uświęca, aby nienaruszony duch wasz, dusza i ciało bez zarzutu zachowały się na przyjście naszego Pana Jezusa Chrystusa.
\vfill
\begin{center}
    \includegraphics[width=340px]{Ryciny/Mszał/pelikan.png}
\end{center}
\vfill\\
\clearpage
\section{Piątek}
% Wołam do Ciebie, Panie, * we dnie i w nocy.
\gregorioscore{GABC/Antyfony/pt_z_inc.gabc}
\subsection*{Psalm 88}
\begin{center}
    \parbox{8cm}{
        \begin{tabbing}
            \hspace{4mm} \= \>\kill
            \color{gray}Panie, mój Boże, wołam do Ciebie we dnie,~*\\
            żalę się przed Tobą w nocy.\\
            \>Niech dotrze do Ciebie moja modlitwa,~*\\
            \>nakłoń ucha na moje wołanie.\\
            Bo moja dusza nieszczęściem jest przepełniona,~*\\
            a życie moje zbliża się do Otchłani.\\
            \>Zaliczono mnie do grona idących do grobu,~*\\
            \>stałem się jak mąż pozbawiony siły.\\
            Między zmarłymi jest moje posłanie,~*\\
            tak jak poległych, którzy leżą w grobach,\\
            \>O których Ty już więcej nie pamiętasz,~*\\
            \>nad którymi nie roztaczasz już opieki.\\
            Strąciłeś mnie w otchłań najgłębszą,~*\\
            w mrok i przepaść.\\
            \>Twój gniew mnie przygniata,~*\\
            \>spiętrzyły się nade mną jego fale.\\
            Odsunąłeś ode mnie wszystkich przyjaciół, †\\
            wzbudziłeś w nich do mnie odrazę,~*\\
            znalazłem się w więzieniu bez wyjścia.\\
            \>Moje oczy słabną w nieszczęściu, †\\
            \>każdego dnia wołam do Ciebie, Panie,~*\\
            \>ręce do Ciebie wyciągam.\\
            Czy uczynisz cud dla umarłych?~*\\
            Czy wstaną cienie, by wielbić Ciebie?\\
        \end{tabbing}}
\end{center}
\begin{center}
    \parbox{8cm}{
        \begin{tabbing}
            \hspace{4mm} \= \>\kill
            \>Czy to w grobach sławi się Twoją łaskę,~*\\
            \>a wierność Twoją w miejscu zagłady?\\
            Czy Twoje cuda widzi się w ciemnościach,~*\\
            a sprawiedliwość w krainie zapomnienia?\\
            \>A ja wołam do Ciebie, Panie,~*\\
            \>niech nad ranem dotrze do Ciebie moja modlitwa.\\
            Czemu odrzucasz mnie, Panie,~*\\
            i ukrywasz swoje oblicze przede mną?\\
            \>Cierpię biedę i od młodości stoję na progu śmierci,~*\\
            \>dźwigałem Twoją grozę i osłabłem.\\
            Przewalił się nade mną płomień Twego gniewu~*\\
            i złamały mnie Twoje groźby.\\
            \>Zewsząd mnie otoczyły jak fale powodzi~*\\
            \>i topią mnie w jednym momencie.\\
            Odsunąłeś ode mnie przyjaciół i towarzyszy,~*\\
            tylko ciemności mieszkają ze mną.\\
            \>Chwała Ojcu i Synowi,~*\\
            \>i Duchowi Świętemu.\\
            Jak była na początku, teraz i zawsze,~*\\
            i na wieki wieków. Amen.
        \end{tabbing}}
\end{center}
\gregorioscore{GABC/Antyfony/pt.gabc}

\subsection*{Czytanie}
% Ty jesteś wśród nas, Panie, a Twoje imię zostało wezwane nad nami, nie opuszczaj nas, Panie, nasz Boże.\\
\gregorioscore{GABC/Czytania/pt.gabc}

\chapter{Procesja}
\label{ch:procesja}
\section*{Procesja}

\gresetinitiallines{1}
\comm{Idąc w procesji do Najświętszej Maryi Panny, śpiewa się następującą antyfonę:}\\
\gregorioscore{GABC/Procesja/salve.gabc}\\
\gresetinitiallines{0}
\comm{Po antyfonie następuje dialog i oracja:}\\
\gregorioscore{GABC/Procesja/salve_dialog.gabc}\\
\gregorioscore{GABC/Procesja/salve_oracja.gabc}\\

\gresetinitiallines{1}
\comm{Wracając do chóru, śpiewa się następującą antyfonę do św. Dominika:}\\
\gregorioscore{GABC/Procesja/lumen.gabc}\\
\gresetinitiallines{0}
\comm{Po antyfonie następuje dialog i oracja:}\\
\gregorioscore{GABC/Procesja/lumen_dialog.gabc}\\
\gregorioscore{GABC/Procesja/lumen_oracja.gabc}\\

\gresetinitiallines{1}
\comm{Wedle zwyczaju prowincji Polskiej w środę po Salve zamiast "O Lumen" śpiewana jest antyfona do św. Jacka:}\\
\gregorioscore{GABC/Procesja/ave_florum.gabc}\\
\gresetinitiallines{0}
\comm{Po antyfonie następuje dialog i oracja:}\\
\gregorioscore{GABC/Procesja/ave_florum_dialog.gabc}\\
\gregorioscore{GABC/Procesja/ave_florum_oracja.gabc}\\



\thispagestyle{empty}
\color{white}\tiny
-
\vfill
\begin{center}
    \includegraphics[width=170px]{Ryciny/Mszał/nmp4.jpg}
\end{center}
\vfill
\color{black}\normalsize
\clearpage



\chapter{Antyfony końcowe}
\label{ch:antyfony_koncowe}
\grechangestaffsize{18}
\gresetinitiallines{1}

\section{Ave Regina Caelorum\\\small{wersja rzymska}}
\label{nt:ave_regina_caelorum}
\gregorioscore{GABC/Antyfony_koncowe/ave_regina_caelorum.gabc}
\vfill
\begin{center}
    \includegraphics[width=340px]{Ryciny/Mszał/ukoronowanie.jpg}
\end{center}
\clearpage

\section{Ave Regina Caelorum\\\small{wersja dominikańska}}
\label{nt:ave_regina_caelorum_op}
\gregorioscore{GABC/Antyfony_koncowe/ave_regina_caelorum_op.gabc}
\vfill
\begin{center}
    \includegraphics[width=185px]{Ryciny/Mszał/nmp_roza.jpg}
\end{center}
\clearpage

\section{Alma Redemtoris Mater\\\small{wersja rzymska}}
\label{nt:alma_redemtoris_mater}
\gregorioscore{GABC/Antyfony_koncowe/alma.gabc}
\vfill
\begin{center}
    \includegraphics[width=340px]{Ryciny/Mszał/nmp_siedmiu_bolesi.jpg}
\end{center}
\clearpage

\section{Alma Redemtoris Mater\\\small{wersja dominikańska}}
\label{nt:alma_redemtoris_mater_op}
\gregorioscore{GABC/Antyfony_koncowe/alma_op.gabc}
\vfill
\begin{center}
    \includegraphics[width=120px]{Ryciny/Herby/102.jpg}
\end{center}
\clearpage

\section{Inviolata}
\label{nt:inviolata}
\gregorioscore{GABC/Antyfony_koncowe/inviolata_op.gabc}
\vfill
\begin{center}
    \includegraphics[width=180px]{Ryciny/Mszał/aniolek3.jpg}
\end{center}
\clearpage

\section{Recordare}
\label{nt:recordare}
\gregorioscore{GABC/Antyfony_koncowe/recordare.gabc}
\vfill
\begin{center}
    \includegraphics[width=240px]{Ryciny/Mszał/nmp_monogram.jpg}
\end{center}
\clearpage
\section{Sub tuum}
\label{nt:sub_tuum}
\gregorioscore{GABC/Antyfony_koncowe/sub_tuum_op.gabc}

\section{Pie Pater}
\label{nt:pie_pater}
\gregorioscore{GABC/Antyfony_koncowe/pie_pater.gabc}

\section{Magne Pater}
\label{nt:magne_pater}
\gregorioscore{GABC/Antyfony_koncowe/magne_pater.gabc}
\clearpage
% OW - dodać Alleluje i Regina Caeli

\grechangestaffsize{15}

\chapter{Teksty własne}
\label{ch:tekty_wlasne}
\gresetinitiallines{1}
% 22 III - 13 VI
\section*{Oktawa Wielkanocy\\\small Do drugiej niedzieli Wielkanocnej włącznie}
\subsection*{Responsorium}
% Oto dzień, który Pan uczynił, * radujmy się nim i weselmy. <- z Brewiarza.
% Oto dzień, który uczynił Pan, * radujmy się i weselmy w nim. <- własne.
% Tenci jest dzień, który uczynił Pan * radujmy się i weselmy się weń <- z Lwowa.
\gregorioscore{GABC/Wlasne/haec_est.gabc}\vspace{3mm}\noindent
\subsection*{Superantyfona}
\gregorioscore{GABC/Antyfony/alleluja_x4.gabc}\vspace{3mm}\noindent
% \vfill
% \begin{center}
%     \includegraphics[width=340px]{Ryciny/Mszał/agnus_dei10.jpg}
% \end{center}
\begin{center}
    \includegraphics[width=180px]{Ryciny/Mszał/baner_rex_gloria_christi.png}
\end{center}

\clearpage

\section*{Uroczystość Zwiastowania Pańskiego\\\small 25 III}

\subsection*{Hymn}
\comm{Należy zastosować zakończenie hymnu \enquote{Maryjo, Matko wszystkich łask}.}

\subsection*{Superantyfona do psalmu/psalmów}
% Ant. Ecce Virgo concipiet et pariet Filium, et vocabitur nomen eius Emmanuel. \\
% PL: Oto Dziewica pocznie i porodzi Syna, a Jego imię będzie Emmanuel\\
\gregorioscore{GABC/Wlasne/ecce_virgo.gabc}\vspace{3mm}\noindent

\subsection*{Antyfona do kantyku Symeona}
% Ecce ancilla Domini, fiat mihi secundum verbum tuum\\
% PL: Oto niewolnica Pańska, niech mi się stanie według słowa Twego.\\
\gregorioscore{GABC/Wlasne/ecce_ancilla.gabc}\vspace{3mm}\noindent
\clearpage
\gregorioscore{GABC/Nunc_dimittis/VIII_b.gabc}\vspace{3mm}\noindent
\vfill
\gregorioscore{GABC/Wlasne/ecce_ancilla.gabc}\vspace{3mm}\noindent

\section*{Uroczystość Zesłania Ducha Świętego\\\small 50 dni po Wielkanocy}
\subsection*{Hymn}
\comm{Należy zastosować poniższe zakończenie hymnu:}

% Dudum sacrata péctora
% tua replésti gratia,
% dimitte nunc pecca.mina,
% et da quiéta témpora.

% Sit laus Patri cum Filio,
% Sancto simul Paraclito,
% nobisque mittat Filius
% charisma Sancti Spiritus.

% Serca są święte dawno już,
% Boś pełnią łask obdarzył nas;
% Teraz sumienia grzeszne wzrusz,
% Aby pokoju nastał czas.

% Ojcu Synowi niech brzmi hymn
% Oraz Duchowi ze wszech miar,
% A nam niech ześle Boski Syn
% Ducha Świętego wieczny dar. Amen.
\gresetinitiallines{0}
\gregorioscore{GABC/Hymny/jesu_nostra_pen.gabc}\vspace{3mm}\noindent
\gresetinitiallines{1}
\subsection*{Antyfona do kantyku Symeona}
\gregorioscore{GABC/Nunc_dimittis/alleluja_pen.gabc}\vspace{3mm}\noindent
\gregorioscore{GABC/Nunc_dimittis/V.gabc}\vspace{3mm}\noindent
\gregorioscore{GABC/Nunc_dimittis/alleluja_pen.gabc}\vspace{3mm}\noindent

\clearpage

\chapter{Teksty wspólne}
\label{ch:tekty_wspolne}
\gresetinitiallines{1}

\section*{W święta i uroczystości\\Najświętszej Maryi Panny}\\
\subsection*{Hymn}
\comm{Stosuje się zakończenie „Maryjo, Matko wszystkich łask”.}

\subsection*{Superantyfona do psalmu/psalmów}
\comm{Ant.} Dziewico Maryjo, \comm{*} nie urodziła się na świecie podobna Tobie między niewiastami, \comm{/} kwitnąca jako róża, wonna jako lilia, \comm{/} módl się za nami święta Boża\\Rodzicielko.

\subsection*{Antyfona do kantyku Symeona}
\comm{Ant.} Duszą i sercem \comm{*} śpiewajmy Chrystusowi chwałę~\comm{/} W tę świętą uroczystość wielkiej Bożej Rodzicielki \mbox{Maryi.}\\
\comm{Można również użyć antyfony \emph{Sub Tuum} - s. \pageref{nt:sub_tuum}.}

\clearpage




% \clearpage
% \thispagestyle{empty}
% \color{white}\tiny
% -
% \vfill
% \begin{center}
%     \includegraphics[width=180px]{Ryciny/Mszał/alfa-omega.png}
% \end{center}
% \vfill
% \color{black}\normalsize
\clearpage
\begin{center}
    \parbox{10cm}{\tableofcontents}
\end{center}
\clearpage
\thispagestyle{empty}
\color{white}\tiny{-}
\vfill
\begin{center}
    \includegraphics[width=300px,right]{Ryciny/Mszał/agnus_dei7.jpg}
\end{center}
\vfill

\end{document}