\gresetinitiallines{1}
% 4 II - 24 IV
% \section*{Święto świętych Cyryla i Metodego\\\small 14 II}
%     \comm{Można odmówić responsorium \enquote{W połowie życia}.}

% \section*{Święto Katedry św. Piotra, Apostoła\\\small 22 II}
%     \comm{Można odmówić responsorium \enquote{W połowie życia}.}

% \section*{Święto św. Kazimierza\\\small 4 III}
%     \comm{Można odmówić responsorium \enquote{W połowie życia}.}

% \section*{Rocznica Poświęcenia\\Bazyliki pw. Świętej Trójcy w Krakowie\\\small 12 III}
%     \comm{Można odmówić responsorium \enquote{W połowie życia}.}

% \section*{Uroczystość św. Józefa – Oblubieńca NMP\\\small 19 III}
%     \comm{Można odmówić responsorium \enquote{W połowie życia}.}

\section*{Uroczystość Zwiastowania Pańskiego\\\small 25 III}
\subsection*{Hymn}
    \comm{Można zastosować zakończenie \enquote{Maryjo, Matko\\wszystkich łask}.}
\subsection*{Superantyfona do psalmu/psalmów}
% Ant. Ecce Virgo concipiet et pariet Filium, et vocabitur nomen eius Emmanuel. \\
% PL: Oto Dziewica pocznie i porodzi Syna, a Jego imię będzie Emmanuel\\
% \gregorioscore{GABC/Wlasne/ecce_virgo.gabc}\vspace{3mm}\noindent
\comm{Ant.} Oto Panna pocznie i porodzi Syna, \comm{*} i nazwą Go imieniem Emmanuel.
\subsection*{Responsorium}
\comm{Można odmówić responsorium \enquote{W połowie życia}.}

\subsection*{Antyfona do kantyku Symeona}
% Ecce ancilla Domini, fiat mihi secundum verbum tuum\\
% PL: Oto niewolnica Pańska, niech mi się stanie według słowa Twego.\\
\comm{Ant.} Oto ja, służebnica Pańska, \comm{*} niech mi się stanie według słowa Twego.
% \vfill
% \begin{center}
%     \includegraphics[width=150px]{Ryciny/Mszał/nmp4.jpg}
% \end{center}