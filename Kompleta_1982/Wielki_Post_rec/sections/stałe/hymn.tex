\section*{Hymn}
\gresetinitiallines{1}

% Christe, qui lux es et dies,
% noctis tenébras détegis,
% lucisque lumen créderis,
% lumen beatum pnédicans.

% Precamur sancte Domine,
% defénde nos in hac nocte,
% sit nobis in te réquies,
% quiétam noctem tribue.

% Ne gravis somnus irruat,
% nec hostis nos subripiat,
% nec caro illi conséntiens
% nos tibi reos statuat.

% Oculi somnum capiant,
% cor ad te semper vigilet,
% déxtera tua protegat
% famulos, qui te diligunt.

% Defénsor noster aspice,
% insidiantes réprime,
% gubérna tuos famulos,
% quos sanguine mercatus es.

% Meménto nostri, Domine,
% in gravi isto corpore;
% qui es defénsor animre,
% adésto nobis, Domine.

% Présta, Pater omnipotens,
% per Iesum Christum Dominum,
% qui tecum in perpétuum
% regnat cum Sancto Spiritu.
\begin{tabbing}
\hspace{5mm} \= \>\kill
\bf{1.}
\>Chryste, Tyś Światłem jest i Dniem,\\
\>Który rozprasza nocy cień.\\
\>Tyś ze Światłości światła blask,\\
\>Jasności świętej głosisz dzień.\\
\\
\bf{2.}
\>Panie, do Twych pukamy bram,\\
\>Niech nas obroni Twoja moc.\\
\>Naszym pokojem bądź Ty sam\\
\>I daj spokojną, cichą noc.\\
\\
\bf{3.}
\>Spraw, by nas nie zmógł ciężar snu\\
\>I nie oszukał wrogi syn,\\
\>Niech ciało nie ulegnie mu,\\
\>Nas obarczając jarzmem win.\\
\\
\bf{4.}
\>Choć na powieki spada sen,\\
\>Lecz serce czuwa w dzień i w noc.\\
\>Twych wiernych, co kochają Cię\\
\>Niech broni Twej prawicy moc.\\
\end{tabbing}

\comm{W trakcie śpiewania wersu \enquote{Których nabyłeś własną Krwią} klęka się dla uczczenia Męki Pańskiej.}\\

\begin{tabbing}
\hspace{5mm} \= \>\kill
\bf{5.}
\>Obróć, Obrońco, na nas wzrok,\\
\>Stłum czyhających zgraję złą,\\
\>Sług swoich rozrządź każdy krok,\\
\>Których nabyłeś własną Krwią.\\
\\
\bf{6.}
\>Pamiętaj o nas, Panie nasz,\\
\>Gdy nas przygniata ciała znój.\\
\>Obrońcą naszym jesteś Ty,\\
\>Więc ku pomocy zawsze stój.\\
\\
\bf{7.}
\>Ojcze Wszechmocny, prosim Cię,\\
\>W Imię Jezusa usłysz nas,\\
\>Co z Tobą poprzez wieków bieg\\
\>Króluje z Duchem Świętym wraz. Amen.\\
\end{tabbing}

\clearpage

\comm{W uroczystości i święta maryjne można zastosować poniższe zakończenie hymnu:}\\

\begin{tabbing}
\hspace{5mm} \= \>\kill
\bf{7.}
\>Maryjo, Matko wszystkich łask,\\
\>Tyś miłosierdzia matką jest.\\
\>Broń, gdy naciera wroga wrzask,\\
\>i gdy się zbliża życia skon.\\
\\
\bf{8.}
\>Do Ciebie wznosim, Panie, hymn,\\
\>Któryś się z Panny zrodzić chciał,\\
\>z Ojcem i Świętym Duchem Twym,\\
\>Byś wiekuistą chwałę miał. Amen.\\
\end{tabbing}

% \gregorioscore{GABC/Hymny/maria_mater_gratiae.gabc}\vspace{3mm}\noindent

\vfill
\begin{center}
    \includegraphics[width=260px]{Ryciny/Mszał/nmp_nieustajacej_pomocy2.jpg}
\end{center}
\vfill
\clearpage