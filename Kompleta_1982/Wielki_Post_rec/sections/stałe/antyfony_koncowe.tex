% \grechangestaffsize{18}
\gresetinitiallines{1}

\section*{Ave Regina Caelorum\\\small{wersja rzymska}}
\label{nt:ave_regina_caelorum}
\gregorioscore{GABC/Antyfony_koncowe/ave_regina_caelorum.gabc}
\vfill
\begin{center}
    \includegraphics[width=230px]{Ryciny/Mszał/ukoronowanie.jpg}
\end{center}
\clearpage

\section*{Ave Regina Caelorum\\\small{wersja dominikańska}}
\label{nt:ave_regina_caelorum_op}
\gregorioscore{GABC/Antyfony_koncowe/ave_regina_caelorum_op.gabc}
\vfill
\begin{center}
    \includegraphics[width=185px]{Ryciny/Mszał/nmp_roza.jpg}
\end{center}
\clearpage

\section*{Inviolata}
\label{nt:inviolata}
\gregorioscore{GABC/Antyfony_koncowe/inviolata_op.gabc}
\vfill
\begin{center}
    \includegraphics[width=150px]{Ryciny/Mszał/gwiazda_nmp.jpg}
\end{center}
\clearpage

\section*{Alma Redemtoris Mater\\\small{wersja rzymska}}
\label{nt:alma_redemtoris_mater}
\gregorioscore{GABC/Antyfony_koncowe/alma.gabc}
\vfill
\begin{center}
    \includegraphics[width=340px]{Ryciny/Mszał/nmp_siedmiu_bolesi.jpg}
\end{center}
\clearpage

\section*{Alma Redemtoris Mater\\\small{wersja dominikańska}}
\label{nt:alma_redemtoris_mater_op}
\gregorioscore{GABC/Antyfony_koncowe/alma_op.gabc}
\vfill
\begin{center}
    \includegraphics[width=120px]{Ryciny/Herby/143.jpg}
\end{center}
\clearpage

\section*{Ave stella}
\label{nt:ave_stella}
\gregorioscore{GABC/Antyfony_koncowe/ave_stella.gabc}
\vfill
\begin{center}
    \includegraphics[width=230px]{Ryciny/Mszał/nmp_monogram.jpg}
\end{center}
\vfill

\clearpage

\section*{Sub tuum}
\label{nt:sub_tuum}
\normalsize
\comm{Można klęczeć.}
\footnotesize
\gregorioscore{GABC/Antyfony_koncowe/sub_tuum_op.gabc}

\section*{Pie Pater}
\label{nt:pie_pater}
\gregorioscore{GABC/Antyfony_koncowe/pie_pater.gabc}

\vfill
\begin{center}
    \includegraphics[width=70px]{Ryciny/Herby/101.jpg}
\end{center}
\clearpage

\section*{Magne Pater}
\label{nt:magne_pater}
\gregorioscore{GABC/Antyfony_koncowe/magne_pater.gabc}

\section*{Recordare}
\label{nt:recordare}
\gregorioscore{GABC/Antyfony_koncowe/recordare.gabc}

\vfill
\begin{center}
    \includegraphics[width=240px]{Ryciny/Herby/97.jpg}
\end{center}
\clearpage
\section*{Salve Regina}
\label{nt:salve}
\gregorioscore{GABC/Procesja/salve.gabc}\vspace{3mm}\noindent
\subsection*{Zakończenie polskie}
\normalsize
\noindent
\R{Pozwól Cię chwalić Panno Święta.}
\V{Daj mi moc przeciw nieprzyjaciołom Twoim.}
\R{Módlmy się. Prosimy Cię, Panie Boże, dozwól nam, sługom swoim, cieszyć się trwałym zdrowiem duszy i ciała,
a za przyczyną Najświętszej Maryi zawsze Dziewicy,
racz nas uwolnić od doczesnych utrapień i obdarzyć wieczną radością.
Przez Chrystusa Pana naszego.}
\V{Amen.}
\subsection*{Zakończenie łacińskie}
\noindent
\R{Dignare me laudare te, Virgo sacrata.}
\V{Da mihi virtutem contra hostes tuos.}
\R{Oremus. Concéde nos famulos tuos, qaésumus, Domine
Deus, perpétua mentis et corporis salute gaudére
et gloriosa beatae Marire semper Virginis intercessione a prresénti liberari tristitia et aetérna pérfrui laetitia. Per Christum Dominum nostrum.}
\V{Amen.}

\footnotesize
\section*{O lumen}
\label{nt:o_lumen}
\gregorioscore{GABC/Procesja/lumen.gabc}\vspace{3mm}\noindent
\normalsize
\subsection*{Zakończenie polskie}
\noindent
\R{Módl się za nami, święty Ojcze Dominiku.}
\V{Abyśmy się stali godnymi obietnic Chrystusowych.}
\R{Módlmy się.
Spraw, prosimy wszechmogący Boże,
niechaj wstawiennictwo błogosławionego Dominika,
Wyznawcy Twego, Ojca naszego,
wyzwoli nas z jarzma grzechów,
które nas przytłaczają.
Przez Chrystusa Pana naszego.}
\V{Amen.}
\subsection*{Zakończenie łacińskie}
\noindent
\R{Ora pro nobis, beate Pater Dominice.}
\V{Ut digni efficiamur promissionibus Christi.}
\R{Oremus. Concéde, quaésumus, omnipotens Deus ut qui
peccatorum nostrorum pondere prémimur, beati Dominici confessoris tui, Patris nostri, patrocinio sublevémur. Per Christum Dominum nostrum.}
\V{Amen.}

\vfill
\begin{center}
    \includegraphics[width=340px]{Ryciny/Mszał/dominik.jpg}
\end{center}
\vfill
\clearpage

\section*{Ave florum}
\footnotesize
\label{nt:ave_florum}
\gregorioscore{GABC/Procesja/ave_florum.gabc}
\normalsize
\subsection*{Zakończenie polskie}
\noindent
\R{Módl się za nami, święty Jacku.}
\V{Abyśmy się stali godnymi obietnic Chrystusowych.}
\clearpage
\R{Módlmy się.
Boże, któryś błogosławionego Jacka,
wyznawcę Twego,
w krajach różnych narodów świątobliwymi dziełami
i cudów chwałą wsławić raczył, daj nam,
abyśmy za Jego przykładem życie nasze poprawili
i w przeciwnościach pomocą Jego byli bronieni.
Przez Chrystusa Pana naszego.}
\V{Amen.}
\subsection*{Zakończenie łacińskie}
\noindent
\R{Ora pro nobis, beate Hyacinthe.}
\V{Ut digni efficiamur promissionibus Christi.}
\R{Oremus. Deus, qui batum Hyacinthum, Confessorem tuum, in diversis nationum populis operum sanctitate et miraculorum gloria fecisti conspicuum: da nobis ud ejus in melius reformemur exemplis, et in adversis protegamur auxiliis. Per Christum Dominum nostrum.}
\V{Amen.}

% \thispagestyle{empty}
% \color{white}\tiny
% -
% \vfill
% \begin{center}
%     \includegraphics[width=80px]{Ryciny/Mszał/alfa-omega.png}
% \end{center}
% \vfill
% \color{black}\normalsize
\clearpage

\grechangestaffsize{15}