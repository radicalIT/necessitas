\section*{Antyfona do kantyku Symeona}

\subsection*{Przed sobotą po Popielcu}\\
\comm{Ant.} Strzeż nas, Panie, gdy czuwamy, \comm{*} podczas snu nas osłaniaj, \comm{/} abyśmy czuwali z Chrystusem \comm{/} i odpoczywali w pokoju.

\subsection*{Od soboty po Popielcu, aż do soboty przed trzecią niedzielą Wielkiego Postu}\\
\comm{Ant.} Racz czuwać nad nami, \comm{*} wieczny Zbawicielu,~\comm{/} by nas nie pochwycił chytry nieprzyjaciel,~\comm{/} ponieważ Ty stałeś się dla nas nieustannym Wspomożycielem.

\subsection*{Od trzeciej niedzieli Wielkiego Postu, aż do Triduum}\\
\comm{Ant.} O Królu pełen chwały \comm{*} pośród Świętych Twoich.~\comm{/} Czci wszelkiej zawsze godny, a jednak niewypowiedziany.~\comm{/} Tyś między nami jest, Panie,~\comm{/} a Imię Twoje Święte nad nami jest wzywane.~\comm{/} Nie opuszczaj nas, o nasz Boże,~\comm{/} ażebyś w dzień sądu nas umieścić raczył~\comm{/} pośród Świętych i wybranych Twoich.~\comm{/} O~Królu błogosławiony!

\section*{Kantyk Symeona}\\
\begin{tabbing}
    \hspace{4mm} \= \>\kill
    Teraz, o Panie, pozwól odejść swemu słudze w pokoju,~*\\
    według słowa Twojego,\\
    \>Bo moje oczy ujrzały Twoje zbawienie,~*\\
    \>któreś przygotował wobec wszystkich narodów:\\
    Światło na oświecenie pogan~*\\
    i chwałę ludu Twego, Izraela.
\end{tabbing}

\comm{Powtarza się antyfonę.}

\clearpage

