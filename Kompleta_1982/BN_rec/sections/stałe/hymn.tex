\clearpage
\section*{Hymn}
\gresetinitiallines{1}

\begin{tabbing}
\hspace{5mm} \= \>\kill
\bf{1.}
\>Zanim zapadnie światła blask,\\
\>Prosim Cię Stwórco wszechrzeczy,\\
\>Abyś, jak zwykłeś tarczą łask\\
\>Nas jako pasterz dobry strzegł.\\
\\
\bf{2.}
\>Serca nasze, niech w Tobie śnią,\\
\>Ciebie wśród snów niech czują wciąż,\\
\>I wielkie światło, chwałę Twą,\\
\>Niech wielbi śpiewem każdy głos.\\
\\
\bf{3.}
\>Daj nam dziś w łasce zdrowia żyć,\\
\>Odnów gorliwość, co jest w nas,\\
\>Gdy morki nocy mają przyjść,\\
\>Rozjaśnij światłem w każdy czas.\\
\\
\bf{4.}
\>Ojcze wszechmocny prosim Cię,\\
\>W Imię Jezusa usłysz nas,\\
\>Co z Tobą poprzez wieków bieg,\\
\>Króluje z Duchem Świętym wraz. Amen.
\end{tabbing}

\clearpage

\comm{W uroczystości i święta maryjne można zastosować poniższe zakończenie hymnu:}
\label{h:mariae_mater}
\begin{tabbing}
\hspace{5mm} \= \>\kill
\bf{4.}
\>Maryjo, Matko wszystkich łask,\\
\>Tyś miłosierdzia matką jest.\\
\>Broń, gdy naciera wroga wrzask,\\
\>i gdy się zbliża życia skon.\\
\\
\bf{5.}
\>Do Ciebie wznosim, Panie, hymn,\\
\>Któryś się z Panny zrodzić chciał,\\
\>z Ojcem i Świętym Duchem Twym,\\
\>Byś wiekuistą chwałę miał. Amen.
\end{tabbing}
\vfill
\begin{center}
    \includegraphics[width=310px]{Ryciny/Mszał/zwiastowanie.jpg}
\end{center}
\vfill
\clearpage