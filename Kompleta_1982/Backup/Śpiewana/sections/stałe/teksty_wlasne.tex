% 2 części:
% i. Od dnia po święcie Chrztu Pańskiego do wtorku przed środą popielcową
% ii. Od poniedziałku po U. Zesłania Ducha Świętego do I Nieszporów 1szej niedzieli adwentu

\gresetinitiallines{1}

\section*{Święto Ofiarowania Pańskiego\\\small 2 II}
\noindent
\subsection*{Superantyfona do psalmu/psalmów}
% Ant. Sancta Dei Génetrix, Virgo semper Maria,
% intercéde pro nobis ad Dominum Deum nostrum.\\
% PL: Święta Matko Boża, zawsze dziewico Maryjo,
% wstawiaj się za nami u Pana Boga naszego.\\
\gregorioscore{GABC/Wlasne/sancta_dei_genitrix.gabc}\vspace{3mm}\noindent
% str. 106

\subsection*{Antyfona do kantyku Symeona}
% Nunc dimittis, Domine, servum tuum in pace, quia vidérunt oculi mei salutare tuum\\
% PL: Teraz, o Panie, Pozwól odejść swemu słudze w pokoju, według słowa Twojego \comm{†}.\\
\gregorioscore{GABC/Wlasne/nunc_dimittis_inc.gabc}\vspace{3mm}\noindent
\comm{Ciąg dalszy śpiewa się normalnie według tonu VII a - s. \pageref{nt:nunc_dimittis_vii_a}.}\\
\hspace{3mm}
\comm{Na koniec następuje antyfona:}
\gregorioscore{GABC/Wlasne/nunc_dimittis.gabc}\vspace{3mm}\noindent
% s. 108
\clearpage

\section*{Uroczystość Najświętszego\\Ciała i Krwi Chrystusa\\\small Czwartek po uroczystości Najświętszej Trójcy}
\subsection*{Hymn}
<znaleźć + inne zakończenie>
\subsection*{Antyfona do kantyku Symeona}
% Alleluia. Panis quem ego dédero, alleluia, caro mea est pro mundi vita, alleluia, alleluia.\\
% PL: Alleluja. Chleb, który Ja dam, alleluja, jest ciałem moim za żywot świata, alleluja, alleluja.\\
\gregorioscore{GABC/Wlasne/panis_que_ego_dedero.gabc}\vspace{3mm}\noindent
\comm{Kantyk Symeona w tonie V na stronie: \pageref{nt:nunc_dimittis_v}.}

\section*{Uroczystość\\Najświętszego Serca Pana Jezusa\\\small Piątek po oktawie Bożego Ciała}
\subsection*{Hymn}
<znaleźć melodię>
\subsection*{Antyfona do kantyku Symeona}
Alleluia. Hauriétis in gaudio, alleluia, de fontibus Salvatoris, alleluia, alleluia.\\
PL: Alleluja. Będziecie czerpać z radością, alleluja, ze źródeł Zbawiciela.\\

\section*{Uroczystość\\Wszystkich Świętych\\\small 1 XI}
% \subsection*{Hymn}
\gregorioscore{GABC/Hymny/omnium_sanctorum.gabc}\vspace{3mm}\noindent
% Sprawdzić - s. 161