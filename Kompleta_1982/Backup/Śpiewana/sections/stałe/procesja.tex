\begin{center}
\parbox{10cm}{
    \setstretch{1.25}
    Niemal od samego początku dominikanie przywiązywali do komplety wielką wagę. Przyjęli bogatą, zróżnicowaną formę tej części oficjum i dodali do niej uroczystą i imponującą ceremonię: procesję Salve Regina.

    \hspace*{2em}W jaki sposób ta procesja została ustanowiona? W lipcu 1221 roku Jordan został mianowany pierwszym prowincjałem nowo założonej prowincji Lombardii. Jordan stwierdza, że wyruszył do Bolonii, chcąc zobaczyć św. Dominika, ale zanim dotarł do tego miasta, święty Założyciel umarł.
    
    \hspace*{2em}Jordan opowiada dalej, że w Bolonii był w tym czasie pewien brat Bernard, który został opętany przez złego ducha i doznawał od niego ciężkich udręk. Jego szaleństwa, trwające w dzień i w nocy, były tak wielkie, że zakłócało to życie całej wspólnoty; i nawet sam Jordan nie był bezpieczny od ataków złego ducha.
    
    \hspace*{2em}I ciągnie dalej: „To tak straszne opętanie brata Bernarda było pierwszą okazją, która skłoniła nas do ustanowienia w Bolonii śpiewu antyfony Salve Regina po komplecie. Ten pobożny i zbawienny zwyczaj zaczął się z tego domu rozszerzać na całą Prowincję Lombardii, aż przyjął się w całym zakonie”.

    \hspace*{2em}Jaka była data tej innowacji? Święty Dominik zmarł 6 sierpnia 1221 roku. Praktyka ta musiała zatem zostać zapoczątkowana w tym samym roku. Tak więc ta tradycja trwa już ponad 8 wieków, z tą różnicą, że teraz zwyczajowo odbywa się po ostatniej wspólnej godzinie brewiarza, tj. po nieszporach.
}
\end{center}
\clearpage
\gresetinitiallines{1}
\gregorioscore{GABC/Procesja/salve.gabc}\vspace{3mm}\noindent
\gresetinitiallines{0}
\comm{Po antyfonie następuje dialog:}\\
\gregorioscore{GABC/Procesja/salve_dialog.gabc}\vspace{3mm}\noindent
\comm{oraz oracja:}
\gregorioscore{GABC/Procesja/salve_oracja.gabc}\vspace{3mm}\noindent
\vfill

\clearpage
\gresetinitiallines{1}
\comm{Wracając do chóru, śpiewa się następującą antyfonę do św. Dominika:}\\
\gregorioscore{GABC/Procesja/lumen.gabc}\vspace{3mm}\noindent
\gresetinitiallines{0}
\comm{Po antyfonie następuje dialog:}
\gregorioscore{GABC/Procesja/lumen_dialog.gabc}\vspace{3mm}\noindent

\vfill
\begin{center}
    \includegraphics[width=80px]{Ryciny/Herby/03[1].jpg}
\end{center}

\clearpage
\comm{oraz oracja:}
\gregorioscore{GABC/Procesja/lumen_oracja.gabc}\vspace{3mm}\noindent

\vfill
\begin{center}
    \includegraphics[width=285px]{Ryciny/Mszał/dominik.jpg}
\end{center}
\vfill

\clearpage
\gresetinitiallines{1}
\comm{Wedle zwyczaju prowincji Polskiej w środę po Salve zamiast "O Lumen" śpiewana jest antyfona do św. Jacka:}\\
\gregorioscore{GABC/Procesja/ave_florum.gabc}\vspace{3mm}\noindent
\gresetinitiallines{0}
\comm{Po antyfonie następuje dialog:}\\
\gregorioscore{GABC/Procesja/ave_florum_dialog.gabc}\vspace{3mm}\noindent
\clearpage
\comm{oraz oracja:}\\
\gregorioscore{GABC/Procesja/ave_florum_oracja.gabc}\vspace{3mm}\noindent

\vfill
\begin{center}
    \includegraphics[width=230px]{Ryciny/Mszał/01-flowers.jpg}
\end{center}