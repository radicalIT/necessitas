\gresetinitiallines{1}

\section*{W święta i uroczystości\\Najświętszej Maryi Panny}\\
\subsection*{Zakończenie hymnu}
\comm{Można zastosować poniższe zakończenie hymnu:}\\
Maria, mater gratire,
mater misericordire,
tu nos ab hoste protege
et bora mortis suscipe.
Gloria tibi, Domine,
qui natus es de Virgine,
cum Patre et Sancto Spiritu,
in sempitérna si.écula.
\subsection*{Superantyfona do psalmu/psalmów}
Ant. Virgo Maria, non est tibi similis nata in mundo inter mulferes, florens ut rosa, fragrans sicut lilium: ora pro nobis, sancta Dei Génetrix.\\ 
PL: Maryjo Dziewico, nie ma wśród kobiet nikogo podobnego tobie na świecie, kwitnącej jak róża, pachnącej jak lilia: módl się za nami, święta Matko Boża\\
% str. 112

\subsection*{Antyfona do kantyku Symeona}
Corde et animo Christo canamus gloriam in hac sacra sollemnitate praecélsae Genetricis Dei Mariae\\
PL: Z sercem i duszą śpiewajmy Chrystusowi chwałę w tej świętej uroczystości, aby uczcić Matkę Bożą Marię.\\
% s. 96
\comm{Można również użyć antyfony Sub Tuum - s. \pageref{nt:sub_tuum}.}
\clearpage
\section*{Hymny}

\subsection*{We wspomnienia dowolne,\\\small{które nie posiadają własnego śpiewu metrycznego}}
\label{nt:hymn_v}
\gregorioscore{GABC/Hymny/V.gabc}
\clearpage

\subsection*{We wspomnienia obowiązkowe,\\\small{które nie posiadają własnego śpiewu metrycznego}}
\label{nt:hymn_iv}
\gregorioscore{GABC/Hymny/IV.gabc}
\clearpage

\subsection*{W święta,\\\small{które nie posiadają własnego śpiewu metrycznego}}
\label{nt:hymn_iii}
\gregorioscore{GABC/Hymny/III.gabc}
\clearpage

\subsection*{W uroczystości jednego Aposotoła,\\\small{które nie posiadają własnego śpiewu metrycznego}}
\label{nt:hymn_ii}
\gregorioscore{GABC/Hymny/II.gabc}
\clearpage

\subsection*{W uroczystości,\\\small{które nie posiadają własnego śpiewu metrycznego}}
\label{nt:hymn_i}
\gregorioscore{GABC/Hymny/I.gabc}
\clearpage