\clearpage
\section*{Hymn}
\gresetinitiallines{1}

\begin{tabbing}
\hspace{5mm} \= \>\kill
\bf{1.}
\>Zanim zapadnie światła blask,\\
\>Prosim Cię Stwórco wszechrzeczy,\\
\>Abyś, jak zwykłeś tarczą łask\\
\>Nas jako pasterz dobry strzegł.
\\
\bf{2.}
\>Serca nasze, niech w Tobie śnią,\\
\>Ciebie wśród snów niech czują wciąż,\\
\>I wielkie światło, chwałę Twą\\
\>Niech wielbi śpiewem każdy głos
\\
\bf{3.}
\>Daj nam dziś w łasce zdrowia żyć,\\
\>Odnów gorliwość, co jest w nas,\\
\>Gdy morki nocy mają przyjść,\\
\>Rozjaśnij światłem w każdy czas.
\\
\bf{4.}
\>Ojcze wszechmocny prosim Cię,\\
\>W Imię Jezusa usłysz nas,\\
\>Co z Tobą poprzez wieków bieg,\\
\>Króluje z Duchem Świętym wraz. Amen.\\
\end{tabbing}

% \vfill
% \begin{center}
%     \includegraphics[width=160px]{Ryciny/Mszał/baner5.jpg}
% \end{center}
\clearpage

\comm{W uroczystości i święta maryjne można zastosować poniższe zakończenie hymnu:}
\begin{tabbing}
\hspace{5mm} \= \>\kill
\bf{4.}
\>Maryjo, Matko wszystkich łask,\\
\>Tyś miłosierdzia matką jest.\\
\>Broń, gdy naciera wroga wrzask,\\
\>i gdy się zbliża życia skon.\\
\\
\bf{5.}
\>Do Ciebie wznosim, Panie, hymn,\\
\>Któryś się z Panny zrodzić chciał,\\
\>z Ojcem i Świętym Duchem Twym,\\
\>Byś wiekuistą chwałę miał. Amen.
\end{tabbing}
\comm{Wówczas stosuje się następującą antyfonę do psalmu:}
% \subsection*{Superantyfona do psalmu/psalmów}
\comm{Ant.} Dziewico Maryjo, \comm{*} nie urodziła się na świecie podobna Tobie między niewiastami, \comm{/} kwitnąca jako róża, wonna jako lilia, \comm{/} módl się za nami święta Boża\\Rodzicielko.\\
\comm{Oraz do kantyku Symeona:}\\
\comm{Ant.} Duszą i sercem \comm{*} śpiewajmy Chrystusowi chwałę~\comm{/} W tę świętą uroczystość wielkiej Bożej Rodzicielki\\Maryji.\\
\comm{Można również użyć antyfony \emph{Sub Tuum} - s. \pageref{nt:sub_tuum}.}
% \vfill
% \begin{center}
%     \includegraphics[width=260px]{Ryciny/Mszał/nmp_nieustajacej_pomocy2.jpg}
% \end{center}
% \vfill
\clearpage