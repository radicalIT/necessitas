\gresetinitiallines{1}
\section*{Oktawa Wielkanocy\\\small Do drugiej niedzieli Wielkanocnej włącznie}
\subsection*{Responsorium}
Oto dzień, który uczynił Pan, \comm{*} radujmy się i weselmy w nim.

\subsection*{Superantyfona}
\comm{Ant.} Alleluja, \comm{*} alleluja, alleluja, alleluja.

\vfill
\begin{center}
    \includegraphics[width=230px]{Ryciny/Mszał/oltarz_2.png}
\end{center}
\vfill
\clearpage
\section*{Uroczystość Zwiastowania Pańskiego\\\small 25 III}
\subsection*{Zakończenie hymnu}
\begin{tabbing}
\hspace{5mm} \= \>\kill
\bf{7.}
\>Maryjo, Matko wszystkich łask,\\
\>Tyś miłosierdzia matką jest.\\
\>Broń, gdy naciera wroga wrzask,\\
\>i gdy się zbliża życia skon.\\
\\
\bf{8.}
\>Do Ciebie wznosim, Panie, hymn,\\
\>Któryś się z Panny zrodzić chciał,\\
\>z Ojcem i Świętym Duchem Twym,\\
\>Byś wiekuistą chwałę miał. Amen.
\end{tabbing}
\subsection*{Superantyfona do psalmu/psalmów}
\comm{Ant.} Oto Panna pocznie i porodzi Syna, \comm{*} i nazwą Go imieniem Emmanuel, alleluja, alleluja.

\subsection*{Antyfona do kantyku Symeona}
\comm{Ant.} Oto ja, służebnica Pańska, \comm{*} niech mi się stanie według słowa Twego, alleluja.

\clearpage
\section*{Uroczystość Zesłania Ducha Świętego\\\small 50 dni po Wielkanocy}
\subsection*{Zakończenie hymnu}
\begin{tabbing}
    \hspace{5mm} \= \>\kill
    \bf{5.}
    \>Serca są święte dawno już,\\
    \>Boś pełnią łask obdarzył nas;\\
    \>Teraz sumienia grzeszne wzrusz,\\
    \>Aby pokoju nastał czas.\\
    \\
    \bf{6.}
    \>Ojcu Synowi niech brzmi hymn\\
    \>Oraz Duchowi ze wszech miar,\\
    \>A nam niech ześle Boski Syn\\
    \>Ducha Świętego wieczny dar. Amen.
\end{tabbing}

\subsection*{Antyfona do kantyku Symeona}
\comm{Ant.} Alleluja. Duch Pocieszyciel, alleluja \comm{*} wszystkiego was nauczy, \comm{/} alleluja, alleluja.
\vfill
\begin{center}
    \includegraphics[width=70px]{Ryciny/Mszał/duch_sw.png}
\end{center}
\vfill