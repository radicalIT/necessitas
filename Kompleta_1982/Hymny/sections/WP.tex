\section{Wielki Post}
\subsection{Hymn}
\gregorioscore{GABC/Hymny/christe_qui_lux_1.gabc}\vspace{3mm}
\gresetinitiallines{0}
\clearpage\noindent
\begin{center}
    \parbox{10cm}{
        \comm{W trakcie śpiewania wersu \enquote{Których nabyłeś własną Krwią} klęka się dla uczczenia Męki Pańskiej.}
    }
\end{center}

\gregorioscore{GABC/Hymny/christe_qui_lux_2.gabc}\vspace{3mm}\noindent

\vfill
\begin{center}
    \includegraphics[width=160px]{Ryciny/Mszał/baner5.jpg}
\end{center}
\clearpage
\begin{center}
    \parbox{10cm}{
        \comm{W uroczystości i święta maryjne można zastosować poniższe zakończenie hymnu:}\\
    }
\end{center}
\gregorioscore{GABC/Hymny/maria_mater_gratiae_WP.gabc}\vspace{3mm}\noindent\gresetinitiallines{0}

\vfill
\begin{center}
    \includegraphics[width=320px]{Ryciny/Mszał/nmp_nieustajacej_pomocy2.jpg}
\end{center}
\vfill
\clearpage
\subsection{Responsorium}

\subsubsection*{W dni powszednie}
\gregorioscore{GABC/Responsoria/in_pace.gabc}\vspace{3mm}\noindent

\subsubsection*{Po I i II nieszporach niedzieli,\\w uroczystości, święta oraz w Wielkim Tygodniu.}
\gregorioscore{GABC/Responsoria/media_vita.gabc}\vspace{3mm}\noindent

\subsubsection*{Triduum}
\comm{W miejsce responsorium mówi się:}\\
Chrystus stał się dla nas posłuszny, aż do śmierci. I to śmierci krzyżowej.

\subsection{Antyfona do Kantyku Symeona}
\gresetinitiallines{1}
\subsection*{Od soboty po Popielcu, aż do soboty\\przed trzecią niedzielą Wielkiego Postu}
\gregorioscore{GABC/Nunc_dimittis/evigila.gabc}\vspace{3mm}\noindent
\begin{center}
    \parbox{10cm}{
        \setstretch{1.25}
        \comm{Kantyk w tonie IV a znajduje się na s. \pageref{nt:nunc_dimittis_iv_a}.
    }}
\end{center}

\subsection*{Od trzeciej niedzieli Wielkiego Postu,\\aż do Triduum}

\gregorioscore{GABC/Nunc_dimittis/o_rex.gabc}\vspace{3mm}\noindent
\begin{center}
    \parbox{10cm}{
        \setstretch{1.25}
        \comm{Kantyk w tonie III b znajduje się na s. \pageref{nt:nunc_dimittis_iii_b}.
    }}
\end{center}