% \section*{Hymn}
% \begin{center}
%     \parbox{10cm}{
%         \setstretch{1.25}
%         \comm{W okresie zwykłym, gdy nie przypada żadne wspomnienie, święto ani uroczystość stosuje się poniższe melodie hymnu.\\
%         \hspace*{0.5cm} Melodie własne i wspólne są zamieszczone w sekcji \emph{Teksty własne}, \emph{Teksty wspólne} oraz w dodatku z własnymi melodiami hymnów.\\
%         \hspace*{0.5cm} W adwencie nie stosuje się melodii wspólnych, jedynie tę, przedstawioną poniżej oraz melodie własne.}
%     }
% \end{center}
% \clearpage
\gresetinitiallines{1}

% \subsection*{Po I nieszporach niedzieli}
% \label{nt:hymn_i_nieszpory}
% \gregorioscore{GABC/Hymny/i_nieszpory.gabc}

% \subsection*{Po II nieszporach niedzieli}
% \label{nt:hymn_ii_nieszpory}
\gregorioscore{GABC/Hymny/ii_nieszpory.gabc}
\gregorioscore{GABC/Hymny/ii_nieszpory.gabc}

% \subsection*{W dni powszednie}
% \label{nt:hymn_feria}
% \gregorioscore{GABC/Hymny/feria.gabc}

% \subsection*{W czasie adwentu}
% \label{nt:hymn_adwent}
% \gregorioscore{GABC/Hymny/adventus.gabc}

% \gresetinitiallines{0}