\gresetinitiallines{1}

\section*{W święta i uroczystości\\Najświętszej Maryi Panny}\\
\subsection*{Hymn}
\comm{Można zastosować zakończenie hymnu \enquote{Maryjo, Matko wszystkich łask}.}

\subsection*{Superantyfona do psalmu/psalmów}
% Ant. Virgo Maria, non est tibi similis nata in mundo inter mulferes, florens ut rosa, fragrans sicut lilium: ora pro nobis, sancta Dei Génetrix.\\ 
% PL: Maryjo Dziewico, nie ma wśród kobiet nikogo podobnego tobie na świecie, kwitnącej jak róża, pachnącej jak lilia: módl się za nami, święta Matko Boża. Alleluja <- z AI
% vel
% Dziewico Maryjo, nie urodziła się na świecie podobna Tobie między niewiastami, kwitnąca jako róża, wonna jako lilia, módl się za nami święta Boża Rodzicielko. Alleluja <- z Lwowa
% vel
% Dziewico Maryjo, na całym świecie nie narodziła się podobna Tobie między niewiastami – kwitnąca jak róża, pachnąca jak lilia, módl się za nami, Święta Boża Rodzicielko. Alleluja <- Rycerz Niepokalanej
\gregorioscore{GABC/Wlasne/virgo_maria.gabc}\vspace{3mm}\noindent
% str. 112

\subsection*{Antyfona do kantyku Symeona}
% Corde et animo Christo canamus gloriam in hac sacra sollemnitate praecélsae Genetricis Dei Mariae\\
% Z sercem i duszą śpiewajmy Chrystusowi chwałę w tej świętej uroczystości, aby uczcić Matkę Bożą Marię. Alleluja, alleluja <- z AI
% vel
% Duszą i sercem śpiewajmy Chrystusowi Chwałę. W tej świętej uroczystości Bogarodzicy Maryi. Alleluja, alleluja <- z godzinek
% vel
% Z duszy i serca Chrystusowi chwałę śpiewajmy w tę świętą uroczystość wielkiej Bożej Rodzicielki Maryi. Alleluja, alleluja <- z Lwowa
% s. 96
\gregorioscore{GABC/Wlasne/corde_et_animo.gabc}\vspace{3mm}\noindent
\gregorioscore{GABC/Nunc_dimittis/VIII_a.gabc}\vspace{3mm}\noindent
\gregorioscore{GABC/Wlasne/corde_et_animo.gabc}\vspace{3mm}\noindent

\comm{Można również użyć antyfony Sub Tuum - s. \pageref{nt:sub_tuum}.}

\vfill
\begin{center}
    \includegraphics[width=205px]{Ryciny/Mszał/nmp.jpg}
\end{center}
\vfill

\clearpage